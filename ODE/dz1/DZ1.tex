\documentclass{article}
\usepackage{listings}
\usepackage[utf8]{inputenc}
\usepackage[english,croatian]{babel}
\usepackage{amsfonts}
\usepackage{amsmath}
\usepackage{circuitikz}
    \usetikzlibrary{arrows}
\usepackage{enumitem}
\usepackage{graphicx}
\usepackage{float}
\usepackage{listings}
\usepackage{pdfpages}
\usepackage{siunitx}
\usepackage{subfig}
\usepackage{pgf}
\usepackage[a4paper,left=1in,top=1in,right=1in,bottom=1in,nohead]{geometry}
\usepackage{xcolor}
\setlist[enumerate,1]{% (
leftmargin=*, itemsep=12pt, label={\textbf{\arabic*.}}}


\title{Prvi domaći zadatak iz Osnova digitalne elektronike}
\author{Luka Simić, 2019/0368}
\date{}

\begin{document}
    \begin{titlepage}
        \maketitle
        \vspace*{\fill}
        \tableofcontents
    \end{titlepage}

    \section{Zadatak 1}
        \subsection{Postavka}
            Za kolo sa slike 1 koje se za $t < 0$ nalazi u stacionarnom stanju:
            \begin{enumerate}[label=(\alph*)]
                \item Matematički odrediti vremenske dijagrame napona u čvorovima $A$ i $B$, u intervalu $-\tau_{off} < t < 5\tau_{on}$ ako se u trenutku $t = 0$ zatvori prekidač $P$ koji je pre toga bio otvoren.
                \item Verifikovati prethodnu tačku upotrebom PSpice simulacije.
                \item Matematički odrediti vremenske dijagrame napona u čvorovima A i B, u intervalu $-\tau_{on} < t < 5\tau_{off}$ ako se u trenutku $t = 0$ otvori prekidač $P$ koji je pre toga bio zatvoren.
                \item Verifikovati prethodnu tačku upotrebom PSpice simulacije.
            \end{enumerate}
            Napomena: Vremenske konstante $\tau_{off}$, $\tau_{on}$ karakterišu kolo za slučaj otvorenog i zatvorenog prekidača, respektivno. Koristiti nekontrolisane prekidače.
        \subsection{Rešenje}
            \subsubsection{Vremenske konstante}
                Dato kolo možemo naći u dva stanja: sa otvorenim i zatvorenim prekidačem. Ako je prekidač otvoren, možemo da izbacimo granu sa otpornikom sa kim je povezan redno.

                Na osnovu toga računamo konstante $\tau$:
                \begin{itemize}
                    \item $\tau_{off}$: Ako je prekidač otvoren, sa kondenzatora se vidi otpornost od $R || 2R$, pa je $\tau_off = \frac{2}{3}RC = 100ns$.
                    \item $\tau_{on}$: Ako je prekidač zatvoren, sa kondenzatora se vidi otpornost od $R || (R + (R || R)) = R || 1.5R$, pa je $\tau_{on} = 90ns$.
                \end{itemize}
            \subsubsection{Prva varijanta}
                Prvo računamo napon u tačkama $A$ i $B$ dok traje stacionarno stanje u $t < 0$. Na osnovu naponskog razdelnika od tri jednaka otpornika otpornosti $R$ dobijamo:
                \begin{equation}
                    \label{va0min}
                    v_A(0^{-}) = \frac{2R}{3R} 10V = 6.66V
                \end{equation}
                \begin{equation}
                    \label{vb0min}
                    v_B(0^{-}) = \frac{R}{3R} 10V = 3.33V
                \end{equation}
                Kada se prekidač spusti, napon na kondenzatoru ostaje isti (jer napon na kondenzatoru ne može naglo da se promeni) pa iz \eqref{vb0min} dobijamo i:
                \begin{equation}
                    \label{vb0}
                    v_B(0^{+}) = v_B(0^{-}) = 3.33V
                \end{equation}
                Kako bismo izračunali $v_A(0^{+})$, možemo zameniti kondenzator naponskim generatorom koji generiše napon $v_B(0^{+})$, odnosno 3.33V. Pošto sad imamo jednu struju koja protiče od generatora napona od 10V do zamenjenog kondenzatora, tu imamo jedan naponski razdelnik i dobijamo da je napon u tački $A$ tik nakon zatvaranja prekidača jednak:
                \begin{equation}
                    \label{va0}
                    v_A(0^{+}) = 10V - \frac{R}{1.5R} (10V - v_A(0^{+})) = 10V - 4.44V = 5.56V
                \end{equation}
                Posle određenog vremena (uglavnom posle $5\tau_{on}$) kolo ponovo ulazi u stacionarno stanje. U ovakom stanju kondenzator je ponovo otvorena veza, samo je raspored otpornika drugačiji, pa na osnovu novog naponskog razdelnika dobijamo:
                \begin{equation}
                    \label{vainf}
                    v_A(\infty) = \frac{1.5R}{2.5R} 10V = 6V
                \end{equation}
                \begin{equation}
                    \label{vbinf}
                    v_B(\infty) = \frac{1R}{2.5R} 10V = 4V
                \end{equation}
                Iz \eqref{vb0} i \eqref{vbinf} dobijamo:
                $$v_B(t) = \left\{ \begin{array}{ll}
                    3.33V & t < 0 \\
                    4V - 0.66V \cdot e^{-\frac{t}{\tau_{on}}} & t \geq 0
                \end{array}\right.$$
                Iz \eqref{va0} i \eqref{vainf} dobijamo:
                $$v_A(t) = \left\{ \begin{array}{ll}
                    6.66V & t < 0 \\
                    6V - 0.44 \cdot e^{-\frac{t}{\tau_{on}}} & t \geq 0
                \end{array}\right.$$
                PSpice simulacija na slikama \ref{PSpice1aA} i \ref{PSpice1aB} kao i grafici gornjih funkcija na slici \ref{MPL1a} vraćaju slične rezultate.
                \begin{figure}[H]
                    \centering
                    \includegraphics[width=\textwidth,height=\textheight,keepaspectratio]{DZ1-1aA.pdf}
                    \caption{Verifikacija rezultata zadatka 1 pod a, grafik napona u tački $A$.}
                    \label{PSpice1aA}
                \end{figure}
                \begin{figure}[H]
                    \centering
                    \includegraphics[width=\textwidth,height=\textheight,keepaspectratio]{DZ1-1aB.pdf}
                    \caption{Verifikacija rezultata zadatka 1 pod a, grafik napona u tački $B$.}
                    \label{PSpice1aB}
                \end{figure}
                \begin{figure}[H]
                    \begin{center}
                        %% Creator: Matplotlib, PGF backend
%%
%% To include the figure in your LaTeX document, write
%%   \input{<filename>.pgf}
%%
%% Make sure the required packages are loaded in your preamble
%%   \usepackage{pgf}
%%
%% and, on pdftex
%%   \usepackage[utf8]{inputenc}\DeclareUnicodeCharacter{2212}{-}
%%
%% or, on luatex and xetex
%%   \usepackage{unicode-math}
%%
%% Figures using additional raster images can only be included by \input if
%% they are in the same directory as the main LaTeX file. For loading figures
%% from other directories you can use the `import` package
%%   \usepackage{import}
%%
%% and then include the figures with
%%   \import{<path to file>}{<filename>.pgf}
%%
%% Matplotlib used the following preamble
%%   \usepackage{fontspec}
%%   \setmainfont{DejaVuSerif.ttf}[Path=/home/perapisar/.local/lib/python3.9/site-packages/matplotlib/mpl-data/fonts/ttf/]
%%   \setsansfont{DejaVuSans.ttf}[Path=/home/perapisar/.local/lib/python3.9/site-packages/matplotlib/mpl-data/fonts/ttf/]
%%   \setmonofont{DejaVuSansMono.ttf}[Path=/home/perapisar/.local/lib/python3.9/site-packages/matplotlib/mpl-data/fonts/ttf/]
%%
\begingroup%
\makeatletter%
\begin{pgfpicture}%
\pgfpathrectangle{\pgfpointorigin}{\pgfqpoint{6.400000in}{4.800000in}}%
\pgfusepath{use as bounding box, clip}%
\begin{pgfscope}%
\pgfsetbuttcap%
\pgfsetmiterjoin%
\definecolor{currentfill}{rgb}{1.000000,1.000000,1.000000}%
\pgfsetfillcolor{currentfill}%
\pgfsetlinewidth{0.000000pt}%
\definecolor{currentstroke}{rgb}{1.000000,1.000000,1.000000}%
\pgfsetstrokecolor{currentstroke}%
\pgfsetdash{}{0pt}%
\pgfpathmoveto{\pgfqpoint{0.000000in}{0.000000in}}%
\pgfpathlineto{\pgfqpoint{6.400000in}{0.000000in}}%
\pgfpathlineto{\pgfqpoint{6.400000in}{4.800000in}}%
\pgfpathlineto{\pgfqpoint{0.000000in}{4.800000in}}%
\pgfpathclose%
\pgfusepath{fill}%
\end{pgfscope}%
\begin{pgfscope}%
\pgfsetbuttcap%
\pgfsetmiterjoin%
\definecolor{currentfill}{rgb}{1.000000,1.000000,1.000000}%
\pgfsetfillcolor{currentfill}%
\pgfsetlinewidth{0.000000pt}%
\definecolor{currentstroke}{rgb}{0.000000,0.000000,0.000000}%
\pgfsetstrokecolor{currentstroke}%
\pgfsetstrokeopacity{0.000000}%
\pgfsetdash{}{0pt}%
\pgfpathmoveto{\pgfqpoint{0.800000in}{0.528000in}}%
\pgfpathlineto{\pgfqpoint{3.054545in}{0.528000in}}%
\pgfpathlineto{\pgfqpoint{3.054545in}{4.224000in}}%
\pgfpathlineto{\pgfqpoint{0.800000in}{4.224000in}}%
\pgfpathclose%
\pgfusepath{fill}%
\end{pgfscope}%
\begin{pgfscope}%
\pgfsetbuttcap%
\pgfsetroundjoin%
\definecolor{currentfill}{rgb}{0.000000,0.000000,0.000000}%
\pgfsetfillcolor{currentfill}%
\pgfsetlinewidth{0.803000pt}%
\definecolor{currentstroke}{rgb}{0.000000,0.000000,0.000000}%
\pgfsetstrokecolor{currentstroke}%
\pgfsetdash{}{0pt}%
\pgfsys@defobject{currentmarker}{\pgfqpoint{0.000000in}{-0.048611in}}{\pgfqpoint{0.000000in}{0.000000in}}{%
\pgfpathmoveto{\pgfqpoint{0.000000in}{0.000000in}}%
\pgfpathlineto{\pgfqpoint{0.000000in}{-0.048611in}}%
\pgfusepath{stroke,fill}%
}%
\begin{pgfscope}%
\pgfsys@transformshift{0.902479in}{0.528000in}%
\pgfsys@useobject{currentmarker}{}%
\end{pgfscope}%
\end{pgfscope}%
\begin{pgfscope}%
\definecolor{textcolor}{rgb}{0.000000,0.000000,0.000000}%
\pgfsetstrokecolor{textcolor}%
\pgfsetfillcolor{textcolor}%
\pgftext[x=0.902479in,y=0.430778in,,top]{\color{textcolor}\sffamily\fontsize{10.000000}{12.000000}\selectfont -100}%
\end{pgfscope}%
\begin{pgfscope}%
\pgfsetbuttcap%
\pgfsetroundjoin%
\definecolor{currentfill}{rgb}{0.000000,0.000000,0.000000}%
\pgfsetfillcolor{currentfill}%
\pgfsetlinewidth{0.803000pt}%
\definecolor{currentstroke}{rgb}{0.000000,0.000000,0.000000}%
\pgfsetstrokecolor{currentstroke}%
\pgfsetdash{}{0pt}%
\pgfsys@defobject{currentmarker}{\pgfqpoint{0.000000in}{-0.048611in}}{\pgfqpoint{0.000000in}{0.000000in}}{%
\pgfpathmoveto{\pgfqpoint{0.000000in}{0.000000in}}%
\pgfpathlineto{\pgfqpoint{0.000000in}{-0.048611in}}%
\pgfusepath{stroke,fill}%
}%
\begin{pgfscope}%
\pgfsys@transformshift{1.456422in}{0.528000in}%
\pgfsys@useobject{currentmarker}{}%
\end{pgfscope}%
\end{pgfscope}%
\begin{pgfscope}%
\definecolor{textcolor}{rgb}{0.000000,0.000000,0.000000}%
\pgfsetstrokecolor{textcolor}%
\pgfsetfillcolor{textcolor}%
\pgftext[x=1.456422in,y=0.430778in,,top]{\color{textcolor}\sffamily\fontsize{10.000000}{12.000000}\selectfont 0}%
\end{pgfscope}%
\begin{pgfscope}%
\pgfsetbuttcap%
\pgfsetroundjoin%
\definecolor{currentfill}{rgb}{0.000000,0.000000,0.000000}%
\pgfsetfillcolor{currentfill}%
\pgfsetlinewidth{0.803000pt}%
\definecolor{currentstroke}{rgb}{0.000000,0.000000,0.000000}%
\pgfsetstrokecolor{currentstroke}%
\pgfsetdash{}{0pt}%
\pgfsys@defobject{currentmarker}{\pgfqpoint{0.000000in}{-0.048611in}}{\pgfqpoint{0.000000in}{0.000000in}}{%
\pgfpathmoveto{\pgfqpoint{0.000000in}{0.000000in}}%
\pgfpathlineto{\pgfqpoint{0.000000in}{-0.048611in}}%
\pgfusepath{stroke,fill}%
}%
\begin{pgfscope}%
\pgfsys@transformshift{2.010364in}{0.528000in}%
\pgfsys@useobject{currentmarker}{}%
\end{pgfscope}%
\end{pgfscope}%
\begin{pgfscope}%
\definecolor{textcolor}{rgb}{0.000000,0.000000,0.000000}%
\pgfsetstrokecolor{textcolor}%
\pgfsetfillcolor{textcolor}%
\pgftext[x=2.010364in,y=0.430778in,,top]{\color{textcolor}\sffamily\fontsize{10.000000}{12.000000}\selectfont 100}%
\end{pgfscope}%
\begin{pgfscope}%
\pgfsetbuttcap%
\pgfsetroundjoin%
\definecolor{currentfill}{rgb}{0.000000,0.000000,0.000000}%
\pgfsetfillcolor{currentfill}%
\pgfsetlinewidth{0.803000pt}%
\definecolor{currentstroke}{rgb}{0.000000,0.000000,0.000000}%
\pgfsetstrokecolor{currentstroke}%
\pgfsetdash{}{0pt}%
\pgfsys@defobject{currentmarker}{\pgfqpoint{0.000000in}{-0.048611in}}{\pgfqpoint{0.000000in}{0.000000in}}{%
\pgfpathmoveto{\pgfqpoint{0.000000in}{0.000000in}}%
\pgfpathlineto{\pgfqpoint{0.000000in}{-0.048611in}}%
\pgfusepath{stroke,fill}%
}%
\begin{pgfscope}%
\pgfsys@transformshift{2.564306in}{0.528000in}%
\pgfsys@useobject{currentmarker}{}%
\end{pgfscope}%
\end{pgfscope}%
\begin{pgfscope}%
\definecolor{textcolor}{rgb}{0.000000,0.000000,0.000000}%
\pgfsetstrokecolor{textcolor}%
\pgfsetfillcolor{textcolor}%
\pgftext[x=2.564306in,y=0.430778in,,top]{\color{textcolor}\sffamily\fontsize{10.000000}{12.000000}\selectfont 200}%
\end{pgfscope}%
\begin{pgfscope}%
\pgfsetbuttcap%
\pgfsetroundjoin%
\definecolor{currentfill}{rgb}{0.000000,0.000000,0.000000}%
\pgfsetfillcolor{currentfill}%
\pgfsetlinewidth{0.803000pt}%
\definecolor{currentstroke}{rgb}{0.000000,0.000000,0.000000}%
\pgfsetstrokecolor{currentstroke}%
\pgfsetdash{}{0pt}%
\pgfsys@defobject{currentmarker}{\pgfqpoint{-0.048611in}{0.000000in}}{\pgfqpoint{-0.000000in}{0.000000in}}{%
\pgfpathmoveto{\pgfqpoint{-0.000000in}{0.000000in}}%
\pgfpathlineto{\pgfqpoint{-0.048611in}{0.000000in}}%
\pgfusepath{stroke,fill}%
}%
\begin{pgfscope}%
\pgfsys@transformshift{0.800000in}{0.818182in}%
\pgfsys@useobject{currentmarker}{}%
\end{pgfscope}%
\end{pgfscope}%
\begin{pgfscope}%
\definecolor{textcolor}{rgb}{0.000000,0.000000,0.000000}%
\pgfsetstrokecolor{textcolor}%
\pgfsetfillcolor{textcolor}%
\pgftext[x=0.481898in, y=0.765420in, left, base]{\color{textcolor}\sffamily\fontsize{10.000000}{12.000000}\selectfont 5.6}%
\end{pgfscope}%
\begin{pgfscope}%
\pgfsetbuttcap%
\pgfsetroundjoin%
\definecolor{currentfill}{rgb}{0.000000,0.000000,0.000000}%
\pgfsetfillcolor{currentfill}%
\pgfsetlinewidth{0.803000pt}%
\definecolor{currentstroke}{rgb}{0.000000,0.000000,0.000000}%
\pgfsetstrokecolor{currentstroke}%
\pgfsetdash{}{0pt}%
\pgfsys@defobject{currentmarker}{\pgfqpoint{-0.048611in}{0.000000in}}{\pgfqpoint{-0.000000in}{0.000000in}}{%
\pgfpathmoveto{\pgfqpoint{-0.000000in}{0.000000in}}%
\pgfpathlineto{\pgfqpoint{-0.048611in}{0.000000in}}%
\pgfusepath{stroke,fill}%
}%
\begin{pgfscope}%
\pgfsys@transformshift{0.800000in}{1.429091in}%
\pgfsys@useobject{currentmarker}{}%
\end{pgfscope}%
\end{pgfscope}%
\begin{pgfscope}%
\definecolor{textcolor}{rgb}{0.000000,0.000000,0.000000}%
\pgfsetstrokecolor{textcolor}%
\pgfsetfillcolor{textcolor}%
\pgftext[x=0.481898in, y=1.376329in, left, base]{\color{textcolor}\sffamily\fontsize{10.000000}{12.000000}\selectfont 5.8}%
\end{pgfscope}%
\begin{pgfscope}%
\pgfsetbuttcap%
\pgfsetroundjoin%
\definecolor{currentfill}{rgb}{0.000000,0.000000,0.000000}%
\pgfsetfillcolor{currentfill}%
\pgfsetlinewidth{0.803000pt}%
\definecolor{currentstroke}{rgb}{0.000000,0.000000,0.000000}%
\pgfsetstrokecolor{currentstroke}%
\pgfsetdash{}{0pt}%
\pgfsys@defobject{currentmarker}{\pgfqpoint{-0.048611in}{0.000000in}}{\pgfqpoint{-0.000000in}{0.000000in}}{%
\pgfpathmoveto{\pgfqpoint{-0.000000in}{0.000000in}}%
\pgfpathlineto{\pgfqpoint{-0.048611in}{0.000000in}}%
\pgfusepath{stroke,fill}%
}%
\begin{pgfscope}%
\pgfsys@transformshift{0.800000in}{2.040000in}%
\pgfsys@useobject{currentmarker}{}%
\end{pgfscope}%
\end{pgfscope}%
\begin{pgfscope}%
\definecolor{textcolor}{rgb}{0.000000,0.000000,0.000000}%
\pgfsetstrokecolor{textcolor}%
\pgfsetfillcolor{textcolor}%
\pgftext[x=0.481898in, y=1.987238in, left, base]{\color{textcolor}\sffamily\fontsize{10.000000}{12.000000}\selectfont 6.0}%
\end{pgfscope}%
\begin{pgfscope}%
\pgfsetbuttcap%
\pgfsetroundjoin%
\definecolor{currentfill}{rgb}{0.000000,0.000000,0.000000}%
\pgfsetfillcolor{currentfill}%
\pgfsetlinewidth{0.803000pt}%
\definecolor{currentstroke}{rgb}{0.000000,0.000000,0.000000}%
\pgfsetstrokecolor{currentstroke}%
\pgfsetdash{}{0pt}%
\pgfsys@defobject{currentmarker}{\pgfqpoint{-0.048611in}{0.000000in}}{\pgfqpoint{-0.000000in}{0.000000in}}{%
\pgfpathmoveto{\pgfqpoint{-0.000000in}{0.000000in}}%
\pgfpathlineto{\pgfqpoint{-0.048611in}{0.000000in}}%
\pgfusepath{stroke,fill}%
}%
\begin{pgfscope}%
\pgfsys@transformshift{0.800000in}{2.650909in}%
\pgfsys@useobject{currentmarker}{}%
\end{pgfscope}%
\end{pgfscope}%
\begin{pgfscope}%
\definecolor{textcolor}{rgb}{0.000000,0.000000,0.000000}%
\pgfsetstrokecolor{textcolor}%
\pgfsetfillcolor{textcolor}%
\pgftext[x=0.481898in, y=2.598148in, left, base]{\color{textcolor}\sffamily\fontsize{10.000000}{12.000000}\selectfont 6.2}%
\end{pgfscope}%
\begin{pgfscope}%
\pgfsetbuttcap%
\pgfsetroundjoin%
\definecolor{currentfill}{rgb}{0.000000,0.000000,0.000000}%
\pgfsetfillcolor{currentfill}%
\pgfsetlinewidth{0.803000pt}%
\definecolor{currentstroke}{rgb}{0.000000,0.000000,0.000000}%
\pgfsetstrokecolor{currentstroke}%
\pgfsetdash{}{0pt}%
\pgfsys@defobject{currentmarker}{\pgfqpoint{-0.048611in}{0.000000in}}{\pgfqpoint{-0.000000in}{0.000000in}}{%
\pgfpathmoveto{\pgfqpoint{-0.000000in}{0.000000in}}%
\pgfpathlineto{\pgfqpoint{-0.048611in}{0.000000in}}%
\pgfusepath{stroke,fill}%
}%
\begin{pgfscope}%
\pgfsys@transformshift{0.800000in}{3.261818in}%
\pgfsys@useobject{currentmarker}{}%
\end{pgfscope}%
\end{pgfscope}%
\begin{pgfscope}%
\definecolor{textcolor}{rgb}{0.000000,0.000000,0.000000}%
\pgfsetstrokecolor{textcolor}%
\pgfsetfillcolor{textcolor}%
\pgftext[x=0.481898in, y=3.209057in, left, base]{\color{textcolor}\sffamily\fontsize{10.000000}{12.000000}\selectfont 6.4}%
\end{pgfscope}%
\begin{pgfscope}%
\pgfsetbuttcap%
\pgfsetroundjoin%
\definecolor{currentfill}{rgb}{0.000000,0.000000,0.000000}%
\pgfsetfillcolor{currentfill}%
\pgfsetlinewidth{0.803000pt}%
\definecolor{currentstroke}{rgb}{0.000000,0.000000,0.000000}%
\pgfsetstrokecolor{currentstroke}%
\pgfsetdash{}{0pt}%
\pgfsys@defobject{currentmarker}{\pgfqpoint{-0.048611in}{0.000000in}}{\pgfqpoint{-0.000000in}{0.000000in}}{%
\pgfpathmoveto{\pgfqpoint{-0.000000in}{0.000000in}}%
\pgfpathlineto{\pgfqpoint{-0.048611in}{0.000000in}}%
\pgfusepath{stroke,fill}%
}%
\begin{pgfscope}%
\pgfsys@transformshift{0.800000in}{3.872727in}%
\pgfsys@useobject{currentmarker}{}%
\end{pgfscope}%
\end{pgfscope}%
\begin{pgfscope}%
\definecolor{textcolor}{rgb}{0.000000,0.000000,0.000000}%
\pgfsetstrokecolor{textcolor}%
\pgfsetfillcolor{textcolor}%
\pgftext[x=0.481898in, y=3.819966in, left, base]{\color{textcolor}\sffamily\fontsize{10.000000}{12.000000}\selectfont 6.6}%
\end{pgfscope}%
\begin{pgfscope}%
\pgfpathrectangle{\pgfqpoint{0.800000in}{0.528000in}}{\pgfqpoint{2.254545in}{3.696000in}}%
\pgfusepath{clip}%
\pgfsetrectcap%
\pgfsetroundjoin%
\pgfsetlinewidth{1.505625pt}%
\definecolor{currentstroke}{rgb}{0.121569,0.466667,0.705882}%
\pgfsetstrokecolor{currentstroke}%
\pgfsetdash{}{0pt}%
\pgfpathmoveto{\pgfqpoint{0.902479in}{4.056000in}}%
\pgfpathlineto{\pgfqpoint{1.454370in}{4.056000in}}%
\pgfpathlineto{\pgfqpoint{1.456422in}{0.696000in}}%
\pgfpathlineto{\pgfqpoint{1.491300in}{0.786811in}}%
\pgfpathlineto{\pgfqpoint{1.526177in}{0.871486in}}%
\pgfpathlineto{\pgfqpoint{1.561055in}{0.950440in}}%
\pgfpathlineto{\pgfqpoint{1.595933in}{1.024059in}}%
\pgfpathlineto{\pgfqpoint{1.630811in}{1.092704in}}%
\pgfpathlineto{\pgfqpoint{1.665689in}{1.156711in}}%
\pgfpathlineto{\pgfqpoint{1.702618in}{1.219775in}}%
\pgfpathlineto{\pgfqpoint{1.739548in}{1.278337in}}%
\pgfpathlineto{\pgfqpoint{1.776477in}{1.332717in}}%
\pgfpathlineto{\pgfqpoint{1.813407in}{1.383215in}}%
\pgfpathlineto{\pgfqpoint{1.850336in}{1.430108in}}%
\pgfpathlineto{\pgfqpoint{1.889317in}{1.475978in}}%
\pgfpathlineto{\pgfqpoint{1.928299in}{1.518399in}}%
\pgfpathlineto{\pgfqpoint{1.967280in}{1.557629in}}%
\pgfpathlineto{\pgfqpoint{2.008312in}{1.595740in}}%
\pgfpathlineto{\pgfqpoint{2.049345in}{1.630840in}}%
\pgfpathlineto{\pgfqpoint{2.092430in}{1.664715in}}%
\pgfpathlineto{\pgfqpoint{2.135514in}{1.695785in}}%
\pgfpathlineto{\pgfqpoint{2.180650in}{1.725580in}}%
\pgfpathlineto{\pgfqpoint{2.227838in}{1.753975in}}%
\pgfpathlineto{\pgfqpoint{2.275025in}{1.779805in}}%
\pgfpathlineto{\pgfqpoint{2.324265in}{1.804275in}}%
\pgfpathlineto{\pgfqpoint{2.375556in}{1.827321in}}%
\pgfpathlineto{\pgfqpoint{2.428898in}{1.848902in}}%
\pgfpathlineto{\pgfqpoint{2.486344in}{1.869700in}}%
\pgfpathlineto{\pgfqpoint{2.545842in}{1.888858in}}%
\pgfpathlineto{\pgfqpoint{2.609443in}{1.906960in}}%
\pgfpathlineto{\pgfqpoint{2.677147in}{1.923854in}}%
\pgfpathlineto{\pgfqpoint{2.748954in}{1.939434in}}%
\pgfpathlineto{\pgfqpoint{2.824865in}{1.953638in}}%
\pgfpathlineto{\pgfqpoint{2.906930in}{1.966745in}}%
\pgfpathlineto{\pgfqpoint{2.952066in}{1.973086in}}%
\pgfpathlineto{\pgfqpoint{2.952066in}{1.973086in}}%
\pgfusepath{stroke}%
\end{pgfscope}%
\begin{pgfscope}%
\pgfsetrectcap%
\pgfsetmiterjoin%
\pgfsetlinewidth{0.803000pt}%
\definecolor{currentstroke}{rgb}{0.000000,0.000000,0.000000}%
\pgfsetstrokecolor{currentstroke}%
\pgfsetdash{}{0pt}%
\pgfpathmoveto{\pgfqpoint{0.800000in}{0.528000in}}%
\pgfpathlineto{\pgfqpoint{0.800000in}{4.224000in}}%
\pgfusepath{stroke}%
\end{pgfscope}%
\begin{pgfscope}%
\pgfsetrectcap%
\pgfsetmiterjoin%
\pgfsetlinewidth{0.803000pt}%
\definecolor{currentstroke}{rgb}{0.000000,0.000000,0.000000}%
\pgfsetstrokecolor{currentstroke}%
\pgfsetdash{}{0pt}%
\pgfpathmoveto{\pgfqpoint{3.054545in}{0.528000in}}%
\pgfpathlineto{\pgfqpoint{3.054545in}{4.224000in}}%
\pgfusepath{stroke}%
\end{pgfscope}%
\begin{pgfscope}%
\pgfsetrectcap%
\pgfsetmiterjoin%
\pgfsetlinewidth{0.803000pt}%
\definecolor{currentstroke}{rgb}{0.000000,0.000000,0.000000}%
\pgfsetstrokecolor{currentstroke}%
\pgfsetdash{}{0pt}%
\pgfpathmoveto{\pgfqpoint{0.800000in}{0.528000in}}%
\pgfpathlineto{\pgfqpoint{3.054545in}{0.528000in}}%
\pgfusepath{stroke}%
\end{pgfscope}%
\begin{pgfscope}%
\pgfsetrectcap%
\pgfsetmiterjoin%
\pgfsetlinewidth{0.803000pt}%
\definecolor{currentstroke}{rgb}{0.000000,0.000000,0.000000}%
\pgfsetstrokecolor{currentstroke}%
\pgfsetdash{}{0pt}%
\pgfpathmoveto{\pgfqpoint{0.800000in}{4.224000in}}%
\pgfpathlineto{\pgfqpoint{3.054545in}{4.224000in}}%
\pgfusepath{stroke}%
\end{pgfscope}%
\begin{pgfscope}%
\definecolor{textcolor}{rgb}{0.000000,0.000000,0.000000}%
\pgfsetstrokecolor{textcolor}%
\pgfsetfillcolor{textcolor}%
\pgftext[x=1.927273in,y=4.307333in,,base]{\color{textcolor}\sffamily\fontsize{12.000000}{14.400000}\selectfont \(\displaystyle v_a(t)\)}%
\end{pgfscope}%
\begin{pgfscope}%
\pgfsetbuttcap%
\pgfsetmiterjoin%
\definecolor{currentfill}{rgb}{1.000000,1.000000,1.000000}%
\pgfsetfillcolor{currentfill}%
\pgfsetlinewidth{0.000000pt}%
\definecolor{currentstroke}{rgb}{0.000000,0.000000,0.000000}%
\pgfsetstrokecolor{currentstroke}%
\pgfsetstrokeopacity{0.000000}%
\pgfsetdash{}{0pt}%
\pgfpathmoveto{\pgfqpoint{3.505455in}{0.528000in}}%
\pgfpathlineto{\pgfqpoint{5.760000in}{0.528000in}}%
\pgfpathlineto{\pgfqpoint{5.760000in}{4.224000in}}%
\pgfpathlineto{\pgfqpoint{3.505455in}{4.224000in}}%
\pgfpathclose%
\pgfusepath{fill}%
\end{pgfscope}%
\begin{pgfscope}%
\pgfsetbuttcap%
\pgfsetroundjoin%
\definecolor{currentfill}{rgb}{0.000000,0.000000,0.000000}%
\pgfsetfillcolor{currentfill}%
\pgfsetlinewidth{0.803000pt}%
\definecolor{currentstroke}{rgb}{0.000000,0.000000,0.000000}%
\pgfsetstrokecolor{currentstroke}%
\pgfsetdash{}{0pt}%
\pgfsys@defobject{currentmarker}{\pgfqpoint{0.000000in}{-0.048611in}}{\pgfqpoint{0.000000in}{0.000000in}}{%
\pgfpathmoveto{\pgfqpoint{0.000000in}{0.000000in}}%
\pgfpathlineto{\pgfqpoint{0.000000in}{-0.048611in}}%
\pgfusepath{stroke,fill}%
}%
\begin{pgfscope}%
\pgfsys@transformshift{3.607934in}{0.528000in}%
\pgfsys@useobject{currentmarker}{}%
\end{pgfscope}%
\end{pgfscope}%
\begin{pgfscope}%
\definecolor{textcolor}{rgb}{0.000000,0.000000,0.000000}%
\pgfsetstrokecolor{textcolor}%
\pgfsetfillcolor{textcolor}%
\pgftext[x=3.607934in,y=0.430778in,,top]{\color{textcolor}\sffamily\fontsize{10.000000}{12.000000}\selectfont -100}%
\end{pgfscope}%
\begin{pgfscope}%
\pgfsetbuttcap%
\pgfsetroundjoin%
\definecolor{currentfill}{rgb}{0.000000,0.000000,0.000000}%
\pgfsetfillcolor{currentfill}%
\pgfsetlinewidth{0.803000pt}%
\definecolor{currentstroke}{rgb}{0.000000,0.000000,0.000000}%
\pgfsetstrokecolor{currentstroke}%
\pgfsetdash{}{0pt}%
\pgfsys@defobject{currentmarker}{\pgfqpoint{0.000000in}{-0.048611in}}{\pgfqpoint{0.000000in}{0.000000in}}{%
\pgfpathmoveto{\pgfqpoint{0.000000in}{0.000000in}}%
\pgfpathlineto{\pgfqpoint{0.000000in}{-0.048611in}}%
\pgfusepath{stroke,fill}%
}%
\begin{pgfscope}%
\pgfsys@transformshift{4.161876in}{0.528000in}%
\pgfsys@useobject{currentmarker}{}%
\end{pgfscope}%
\end{pgfscope}%
\begin{pgfscope}%
\definecolor{textcolor}{rgb}{0.000000,0.000000,0.000000}%
\pgfsetstrokecolor{textcolor}%
\pgfsetfillcolor{textcolor}%
\pgftext[x=4.161876in,y=0.430778in,,top]{\color{textcolor}\sffamily\fontsize{10.000000}{12.000000}\selectfont 0}%
\end{pgfscope}%
\begin{pgfscope}%
\pgfsetbuttcap%
\pgfsetroundjoin%
\definecolor{currentfill}{rgb}{0.000000,0.000000,0.000000}%
\pgfsetfillcolor{currentfill}%
\pgfsetlinewidth{0.803000pt}%
\definecolor{currentstroke}{rgb}{0.000000,0.000000,0.000000}%
\pgfsetstrokecolor{currentstroke}%
\pgfsetdash{}{0pt}%
\pgfsys@defobject{currentmarker}{\pgfqpoint{0.000000in}{-0.048611in}}{\pgfqpoint{0.000000in}{0.000000in}}{%
\pgfpathmoveto{\pgfqpoint{0.000000in}{0.000000in}}%
\pgfpathlineto{\pgfqpoint{0.000000in}{-0.048611in}}%
\pgfusepath{stroke,fill}%
}%
\begin{pgfscope}%
\pgfsys@transformshift{4.715819in}{0.528000in}%
\pgfsys@useobject{currentmarker}{}%
\end{pgfscope}%
\end{pgfscope}%
\begin{pgfscope}%
\definecolor{textcolor}{rgb}{0.000000,0.000000,0.000000}%
\pgfsetstrokecolor{textcolor}%
\pgfsetfillcolor{textcolor}%
\pgftext[x=4.715819in,y=0.430778in,,top]{\color{textcolor}\sffamily\fontsize{10.000000}{12.000000}\selectfont 100}%
\end{pgfscope}%
\begin{pgfscope}%
\pgfsetbuttcap%
\pgfsetroundjoin%
\definecolor{currentfill}{rgb}{0.000000,0.000000,0.000000}%
\pgfsetfillcolor{currentfill}%
\pgfsetlinewidth{0.803000pt}%
\definecolor{currentstroke}{rgb}{0.000000,0.000000,0.000000}%
\pgfsetstrokecolor{currentstroke}%
\pgfsetdash{}{0pt}%
\pgfsys@defobject{currentmarker}{\pgfqpoint{0.000000in}{-0.048611in}}{\pgfqpoint{0.000000in}{0.000000in}}{%
\pgfpathmoveto{\pgfqpoint{0.000000in}{0.000000in}}%
\pgfpathlineto{\pgfqpoint{0.000000in}{-0.048611in}}%
\pgfusepath{stroke,fill}%
}%
\begin{pgfscope}%
\pgfsys@transformshift{5.269761in}{0.528000in}%
\pgfsys@useobject{currentmarker}{}%
\end{pgfscope}%
\end{pgfscope}%
\begin{pgfscope}%
\definecolor{textcolor}{rgb}{0.000000,0.000000,0.000000}%
\pgfsetstrokecolor{textcolor}%
\pgfsetfillcolor{textcolor}%
\pgftext[x=5.269761in,y=0.430778in,,top]{\color{textcolor}\sffamily\fontsize{10.000000}{12.000000}\selectfont 200}%
\end{pgfscope}%
\begin{pgfscope}%
\pgfsetbuttcap%
\pgfsetroundjoin%
\definecolor{currentfill}{rgb}{0.000000,0.000000,0.000000}%
\pgfsetfillcolor{currentfill}%
\pgfsetlinewidth{0.803000pt}%
\definecolor{currentstroke}{rgb}{0.000000,0.000000,0.000000}%
\pgfsetstrokecolor{currentstroke}%
\pgfsetdash{}{0pt}%
\pgfsys@defobject{currentmarker}{\pgfqpoint{-0.048611in}{0.000000in}}{\pgfqpoint{-0.000000in}{0.000000in}}{%
\pgfpathmoveto{\pgfqpoint{-0.000000in}{0.000000in}}%
\pgfpathlineto{\pgfqpoint{-0.048611in}{0.000000in}}%
\pgfusepath{stroke,fill}%
}%
\begin{pgfscope}%
\pgfsys@transformshift{3.505455in}{0.537793in}%
\pgfsys@useobject{currentmarker}{}%
\end{pgfscope}%
\end{pgfscope}%
\begin{pgfscope}%
\definecolor{textcolor}{rgb}{0.000000,0.000000,0.000000}%
\pgfsetstrokecolor{textcolor}%
\pgfsetfillcolor{textcolor}%
\pgftext[x=3.187353in, y=0.485032in, left, base]{\color{textcolor}\sffamily\fontsize{10.000000}{12.000000}\selectfont 3.3}%
\end{pgfscope}%
\begin{pgfscope}%
\pgfsetbuttcap%
\pgfsetroundjoin%
\definecolor{currentfill}{rgb}{0.000000,0.000000,0.000000}%
\pgfsetfillcolor{currentfill}%
\pgfsetlinewidth{0.803000pt}%
\definecolor{currentstroke}{rgb}{0.000000,0.000000,0.000000}%
\pgfsetstrokecolor{currentstroke}%
\pgfsetdash{}{0pt}%
\pgfsys@defobject{currentmarker}{\pgfqpoint{-0.048611in}{0.000000in}}{\pgfqpoint{-0.000000in}{0.000000in}}{%
\pgfpathmoveto{\pgfqpoint{-0.000000in}{0.000000in}}%
\pgfpathlineto{\pgfqpoint{-0.048611in}{0.000000in}}%
\pgfusepath{stroke,fill}%
}%
\begin{pgfscope}%
\pgfsys@transformshift{3.505455in}{1.065149in}%
\pgfsys@useobject{currentmarker}{}%
\end{pgfscope}%
\end{pgfscope}%
\begin{pgfscope}%
\definecolor{textcolor}{rgb}{0.000000,0.000000,0.000000}%
\pgfsetstrokecolor{textcolor}%
\pgfsetfillcolor{textcolor}%
\pgftext[x=3.187353in, y=1.012388in, left, base]{\color{textcolor}\sffamily\fontsize{10.000000}{12.000000}\selectfont 3.4}%
\end{pgfscope}%
\begin{pgfscope}%
\pgfsetbuttcap%
\pgfsetroundjoin%
\definecolor{currentfill}{rgb}{0.000000,0.000000,0.000000}%
\pgfsetfillcolor{currentfill}%
\pgfsetlinewidth{0.803000pt}%
\definecolor{currentstroke}{rgb}{0.000000,0.000000,0.000000}%
\pgfsetstrokecolor{currentstroke}%
\pgfsetdash{}{0pt}%
\pgfsys@defobject{currentmarker}{\pgfqpoint{-0.048611in}{0.000000in}}{\pgfqpoint{-0.000000in}{0.000000in}}{%
\pgfpathmoveto{\pgfqpoint{-0.000000in}{0.000000in}}%
\pgfpathlineto{\pgfqpoint{-0.048611in}{0.000000in}}%
\pgfusepath{stroke,fill}%
}%
\begin{pgfscope}%
\pgfsys@transformshift{3.505455in}{1.592506in}%
\pgfsys@useobject{currentmarker}{}%
\end{pgfscope}%
\end{pgfscope}%
\begin{pgfscope}%
\definecolor{textcolor}{rgb}{0.000000,0.000000,0.000000}%
\pgfsetstrokecolor{textcolor}%
\pgfsetfillcolor{textcolor}%
\pgftext[x=3.187353in, y=1.539744in, left, base]{\color{textcolor}\sffamily\fontsize{10.000000}{12.000000}\selectfont 3.5}%
\end{pgfscope}%
\begin{pgfscope}%
\pgfsetbuttcap%
\pgfsetroundjoin%
\definecolor{currentfill}{rgb}{0.000000,0.000000,0.000000}%
\pgfsetfillcolor{currentfill}%
\pgfsetlinewidth{0.803000pt}%
\definecolor{currentstroke}{rgb}{0.000000,0.000000,0.000000}%
\pgfsetstrokecolor{currentstroke}%
\pgfsetdash{}{0pt}%
\pgfsys@defobject{currentmarker}{\pgfqpoint{-0.048611in}{0.000000in}}{\pgfqpoint{-0.000000in}{0.000000in}}{%
\pgfpathmoveto{\pgfqpoint{-0.000000in}{0.000000in}}%
\pgfpathlineto{\pgfqpoint{-0.048611in}{0.000000in}}%
\pgfusepath{stroke,fill}%
}%
\begin{pgfscope}%
\pgfsys@transformshift{3.505455in}{2.119862in}%
\pgfsys@useobject{currentmarker}{}%
\end{pgfscope}%
\end{pgfscope}%
\begin{pgfscope}%
\definecolor{textcolor}{rgb}{0.000000,0.000000,0.000000}%
\pgfsetstrokecolor{textcolor}%
\pgfsetfillcolor{textcolor}%
\pgftext[x=3.187353in, y=2.067100in, left, base]{\color{textcolor}\sffamily\fontsize{10.000000}{12.000000}\selectfont 3.6}%
\end{pgfscope}%
\begin{pgfscope}%
\pgfsetbuttcap%
\pgfsetroundjoin%
\definecolor{currentfill}{rgb}{0.000000,0.000000,0.000000}%
\pgfsetfillcolor{currentfill}%
\pgfsetlinewidth{0.803000pt}%
\definecolor{currentstroke}{rgb}{0.000000,0.000000,0.000000}%
\pgfsetstrokecolor{currentstroke}%
\pgfsetdash{}{0pt}%
\pgfsys@defobject{currentmarker}{\pgfqpoint{-0.048611in}{0.000000in}}{\pgfqpoint{-0.000000in}{0.000000in}}{%
\pgfpathmoveto{\pgfqpoint{-0.000000in}{0.000000in}}%
\pgfpathlineto{\pgfqpoint{-0.048611in}{0.000000in}}%
\pgfusepath{stroke,fill}%
}%
\begin{pgfscope}%
\pgfsys@transformshift{3.505455in}{2.647218in}%
\pgfsys@useobject{currentmarker}{}%
\end{pgfscope}%
\end{pgfscope}%
\begin{pgfscope}%
\definecolor{textcolor}{rgb}{0.000000,0.000000,0.000000}%
\pgfsetstrokecolor{textcolor}%
\pgfsetfillcolor{textcolor}%
\pgftext[x=3.187353in, y=2.594456in, left, base]{\color{textcolor}\sffamily\fontsize{10.000000}{12.000000}\selectfont 3.7}%
\end{pgfscope}%
\begin{pgfscope}%
\pgfsetbuttcap%
\pgfsetroundjoin%
\definecolor{currentfill}{rgb}{0.000000,0.000000,0.000000}%
\pgfsetfillcolor{currentfill}%
\pgfsetlinewidth{0.803000pt}%
\definecolor{currentstroke}{rgb}{0.000000,0.000000,0.000000}%
\pgfsetstrokecolor{currentstroke}%
\pgfsetdash{}{0pt}%
\pgfsys@defobject{currentmarker}{\pgfqpoint{-0.048611in}{0.000000in}}{\pgfqpoint{-0.000000in}{0.000000in}}{%
\pgfpathmoveto{\pgfqpoint{-0.000000in}{0.000000in}}%
\pgfpathlineto{\pgfqpoint{-0.048611in}{0.000000in}}%
\pgfusepath{stroke,fill}%
}%
\begin{pgfscope}%
\pgfsys@transformshift{3.505455in}{3.174574in}%
\pgfsys@useobject{currentmarker}{}%
\end{pgfscope}%
\end{pgfscope}%
\begin{pgfscope}%
\definecolor{textcolor}{rgb}{0.000000,0.000000,0.000000}%
\pgfsetstrokecolor{textcolor}%
\pgfsetfillcolor{textcolor}%
\pgftext[x=3.187353in, y=3.121813in, left, base]{\color{textcolor}\sffamily\fontsize{10.000000}{12.000000}\selectfont 3.8}%
\end{pgfscope}%
\begin{pgfscope}%
\pgfsetbuttcap%
\pgfsetroundjoin%
\definecolor{currentfill}{rgb}{0.000000,0.000000,0.000000}%
\pgfsetfillcolor{currentfill}%
\pgfsetlinewidth{0.803000pt}%
\definecolor{currentstroke}{rgb}{0.000000,0.000000,0.000000}%
\pgfsetstrokecolor{currentstroke}%
\pgfsetdash{}{0pt}%
\pgfsys@defobject{currentmarker}{\pgfqpoint{-0.048611in}{0.000000in}}{\pgfqpoint{-0.000000in}{0.000000in}}{%
\pgfpathmoveto{\pgfqpoint{-0.000000in}{0.000000in}}%
\pgfpathlineto{\pgfqpoint{-0.048611in}{0.000000in}}%
\pgfusepath{stroke,fill}%
}%
\begin{pgfscope}%
\pgfsys@transformshift{3.505455in}{3.701930in}%
\pgfsys@useobject{currentmarker}{}%
\end{pgfscope}%
\end{pgfscope}%
\begin{pgfscope}%
\definecolor{textcolor}{rgb}{0.000000,0.000000,0.000000}%
\pgfsetstrokecolor{textcolor}%
\pgfsetfillcolor{textcolor}%
\pgftext[x=3.187353in, y=3.649169in, left, base]{\color{textcolor}\sffamily\fontsize{10.000000}{12.000000}\selectfont 3.9}%
\end{pgfscope}%
\begin{pgfscope}%
\pgfpathrectangle{\pgfqpoint{3.505455in}{0.528000in}}{\pgfqpoint{2.254545in}{3.696000in}}%
\pgfusepath{clip}%
\pgfsetrectcap%
\pgfsetroundjoin%
\pgfsetlinewidth{1.505625pt}%
\definecolor{currentstroke}{rgb}{0.121569,0.466667,0.705882}%
\pgfsetstrokecolor{currentstroke}%
\pgfsetdash{}{0pt}%
\pgfpathmoveto{\pgfqpoint{3.607934in}{0.696000in}}%
\pgfpathlineto{\pgfqpoint{4.159825in}{0.696000in}}%
\pgfpathlineto{\pgfqpoint{4.161876in}{0.748736in}}%
\pgfpathlineto{\pgfqpoint{4.194702in}{0.970526in}}%
\pgfpathlineto{\pgfqpoint{4.229580in}{1.190713in}}%
\pgfpathlineto{\pgfqpoint{4.264458in}{1.396023in}}%
\pgfpathlineto{\pgfqpoint{4.299336in}{1.587460in}}%
\pgfpathlineto{\pgfqpoint{4.334214in}{1.765963in}}%
\pgfpathlineto{\pgfqpoint{4.369092in}{1.932404in}}%
\pgfpathlineto{\pgfqpoint{4.403970in}{2.087599in}}%
\pgfpathlineto{\pgfqpoint{4.438847in}{2.232308in}}%
\pgfpathlineto{\pgfqpoint{4.473725in}{2.367240in}}%
\pgfpathlineto{\pgfqpoint{4.508603in}{2.493054in}}%
\pgfpathlineto{\pgfqpoint{4.543481in}{2.610368in}}%
\pgfpathlineto{\pgfqpoint{4.578359in}{2.719755in}}%
\pgfpathlineto{\pgfqpoint{4.613237in}{2.821750in}}%
\pgfpathlineto{\pgfqpoint{4.648115in}{2.916854in}}%
\pgfpathlineto{\pgfqpoint{4.682992in}{3.005533in}}%
\pgfpathlineto{\pgfqpoint{4.717870in}{3.088219in}}%
\pgfpathlineto{\pgfqpoint{4.752748in}{3.165318in}}%
\pgfpathlineto{\pgfqpoint{4.787626in}{3.237208in}}%
\pgfpathlineto{\pgfqpoint{4.822504in}{3.304241in}}%
\pgfpathlineto{\pgfqpoint{4.859433in}{3.370286in}}%
\pgfpathlineto{\pgfqpoint{4.896363in}{3.431617in}}%
\pgfpathlineto{\pgfqpoint{4.933292in}{3.488568in}}%
\pgfpathlineto{\pgfqpoint{4.970222in}{3.541453in}}%
\pgfpathlineto{\pgfqpoint{5.007151in}{3.590562in}}%
\pgfpathlineto{\pgfqpoint{5.046132in}{3.638601in}}%
\pgfpathlineto{\pgfqpoint{5.085114in}{3.683027in}}%
\pgfpathlineto{\pgfqpoint{5.124095in}{3.724112in}}%
\pgfpathlineto{\pgfqpoint{5.165127in}{3.764025in}}%
\pgfpathlineto{\pgfqpoint{5.206160in}{3.800784in}}%
\pgfpathlineto{\pgfqpoint{5.247193in}{3.834640in}}%
\pgfpathlineto{\pgfqpoint{5.290277in}{3.867313in}}%
\pgfpathlineto{\pgfqpoint{5.333362in}{3.897281in}}%
\pgfpathlineto{\pgfqpoint{5.378498in}{3.926019in}}%
\pgfpathlineto{\pgfqpoint{5.425686in}{3.953406in}}%
\pgfpathlineto{\pgfqpoint{5.474925in}{3.979351in}}%
\pgfpathlineto{\pgfqpoint{5.526216in}{4.003786in}}%
\pgfpathlineto{\pgfqpoint{5.579558in}{4.026668in}}%
\pgfpathlineto{\pgfqpoint{5.634953in}{4.047976in}}%
\pgfpathlineto{\pgfqpoint{5.657521in}{4.056000in}}%
\pgfpathlineto{\pgfqpoint{5.657521in}{4.056000in}}%
\pgfusepath{stroke}%
\end{pgfscope}%
\begin{pgfscope}%
\pgfsetrectcap%
\pgfsetmiterjoin%
\pgfsetlinewidth{0.803000pt}%
\definecolor{currentstroke}{rgb}{0.000000,0.000000,0.000000}%
\pgfsetstrokecolor{currentstroke}%
\pgfsetdash{}{0pt}%
\pgfpathmoveto{\pgfqpoint{3.505455in}{0.528000in}}%
\pgfpathlineto{\pgfqpoint{3.505455in}{4.224000in}}%
\pgfusepath{stroke}%
\end{pgfscope}%
\begin{pgfscope}%
\pgfsetrectcap%
\pgfsetmiterjoin%
\pgfsetlinewidth{0.803000pt}%
\definecolor{currentstroke}{rgb}{0.000000,0.000000,0.000000}%
\pgfsetstrokecolor{currentstroke}%
\pgfsetdash{}{0pt}%
\pgfpathmoveto{\pgfqpoint{5.760000in}{0.528000in}}%
\pgfpathlineto{\pgfqpoint{5.760000in}{4.224000in}}%
\pgfusepath{stroke}%
\end{pgfscope}%
\begin{pgfscope}%
\pgfsetrectcap%
\pgfsetmiterjoin%
\pgfsetlinewidth{0.803000pt}%
\definecolor{currentstroke}{rgb}{0.000000,0.000000,0.000000}%
\pgfsetstrokecolor{currentstroke}%
\pgfsetdash{}{0pt}%
\pgfpathmoveto{\pgfqpoint{3.505455in}{0.528000in}}%
\pgfpathlineto{\pgfqpoint{5.760000in}{0.528000in}}%
\pgfusepath{stroke}%
\end{pgfscope}%
\begin{pgfscope}%
\pgfsetrectcap%
\pgfsetmiterjoin%
\pgfsetlinewidth{0.803000pt}%
\definecolor{currentstroke}{rgb}{0.000000,0.000000,0.000000}%
\pgfsetstrokecolor{currentstroke}%
\pgfsetdash{}{0pt}%
\pgfpathmoveto{\pgfqpoint{3.505455in}{4.224000in}}%
\pgfpathlineto{\pgfqpoint{5.760000in}{4.224000in}}%
\pgfusepath{stroke}%
\end{pgfscope}%
\begin{pgfscope}%
\definecolor{textcolor}{rgb}{0.000000,0.000000,0.000000}%
\pgfsetstrokecolor{textcolor}%
\pgfsetfillcolor{textcolor}%
\pgftext[x=4.632727in,y=4.307333in,,base]{\color{textcolor}\sffamily\fontsize{12.000000}{14.400000}\selectfont \(\displaystyle v_b(t)\)}%
\end{pgfscope}%
\end{pgfpicture}%
\makeatother%
\endgroup%

                    \end{center}
                    \caption{Grafici izračunatih funkcija $v_a(t)$ i $v_b(t)$ u prvom zadatku pod a.}
                    \label{MPL1a}
                \end{figure}
            \subsubsection{Druga varijanta}
                U drugoj stavci zadatka su iste vremenske konstante, a naponi u stacionarnom stanju su samo obrnuti. Iz \eqref{vainf} i \eqref{vbinf} dobijamo da je napon u stacionarnom stanju za $t < 0$ u tački $A$ jednak 6V a u tački $B$ 4V, dok iz \eqref{va0min} i \eqref{vb0min} dobijamo da je napon u stacionarnom stanju za $t \geq 0$ u tački $A$ jednak 6.66V a u tački $B$ 3.33V. Odatle sledi da je napon u tački $B$ jednak:
                $$v_B(t) = \left\{ \begin{array}{ll}
                    4V & t < 0 \\
                    3.33V + 0.66V \cdot e^{-\frac{t}{\tau_{off}}} & t \geq 0
                \end{array}\right.$$
                Za izračunavanje napona u tački $A$ nam je još potreban napon u toj tački tik nakon otvaranja prekidača. On se kao i u prvoj varijanti može izračunati tako što se kondenzator zameni generatorom napona. Onda imamo naponski razdelnik između 10V i 4V pa dobijamo:
                \begin{equation}
                    \label{va02}
                    v_A(0^{+}) = 10V - \frac{R}{2R} (10V - 4V) = 10V - 3V = 7V
                \end{equation}
                Iz \eqref{va02}, \eqref{vainf} i \eqref{va0min} dobijamo:
                $$v_A(t) = \left\{ \begin{array}{ll}
                    6V & t < 0 \\
                    6.66V + 0.34V \cdot e^{-\frac{t}{\tau_{off}}} & t \geq 0
                \end{array}\right.$$
                PSpice simulacija na slikama \ref{PSpice1bA} i \ref{PSpice1bB} vraćaju slične rezultate kao grafici gornjih funkcija na slici \ref{MPL1b}, sa šemama sa slike \ref{PSpice1}.
                \begin{figure}[H]
                    \centering
                    \includegraphics[width=\textwidth,height=\textheight,keepaspectratio]{DZ1-1bA.pdf}
                    \caption{Verifikacija rezultata zadatka 1 pod b, grafik napona u tački $A$.}
                    \label{PSpice1bA}
                \end{figure}
                \begin{figure}[H]
                    \centering
                    \includegraphics[width=\textwidth,height=\textheight,keepaspectratio]{DZ1-1bB.pdf}
                    \caption{Verifikacija rezultata zadatka 1 pod b, grafik napona u tački $B$.}
                    \label{PSpice1bB}
                \end{figure}
                \begin{figure}[H]
                    \begin{center}
                        %% Creator: Matplotlib, PGF backend
%%
%% To include the figure in your LaTeX document, write
%%   \input{<filename>.pgf}
%%
%% Make sure the required packages are loaded in your preamble
%%   \usepackage{pgf}
%%
%% and, on pdftex
%%   \usepackage[utf8]{inputenc}\DeclareUnicodeCharacter{2212}{-}
%%
%% or, on luatex and xetex
%%   \usepackage{unicode-math}
%%
%% Figures using additional raster images can only be included by \input if
%% they are in the same directory as the main LaTeX file. For loading figures
%% from other directories you can use the `import` package
%%   \usepackage{import}
%%
%% and then include the figures with
%%   \import{<path to file>}{<filename>.pgf}
%%
%% Matplotlib used the following preamble
%%   \usepackage{fontspec}
%%   \setmainfont{DejaVuSerif.ttf}[Path=/home/perapisar/.local/lib/python3.9/site-packages/matplotlib/mpl-data/fonts/ttf/]
%%   \setsansfont{DejaVuSans.ttf}[Path=/home/perapisar/.local/lib/python3.9/site-packages/matplotlib/mpl-data/fonts/ttf/]
%%   \setmonofont{DejaVuSansMono.ttf}[Path=/home/perapisar/.local/lib/python3.9/site-packages/matplotlib/mpl-data/fonts/ttf/]
%%
\begingroup%
\makeatletter%
\begin{pgfpicture}%
\pgfpathrectangle{\pgfpointorigin}{\pgfqpoint{6.400000in}{4.800000in}}%
\pgfusepath{use as bounding box, clip}%
\begin{pgfscope}%
\pgfsetbuttcap%
\pgfsetmiterjoin%
\definecolor{currentfill}{rgb}{1.000000,1.000000,1.000000}%
\pgfsetfillcolor{currentfill}%
\pgfsetlinewidth{0.000000pt}%
\definecolor{currentstroke}{rgb}{1.000000,1.000000,1.000000}%
\pgfsetstrokecolor{currentstroke}%
\pgfsetdash{}{0pt}%
\pgfpathmoveto{\pgfqpoint{0.000000in}{0.000000in}}%
\pgfpathlineto{\pgfqpoint{6.400000in}{0.000000in}}%
\pgfpathlineto{\pgfqpoint{6.400000in}{4.800000in}}%
\pgfpathlineto{\pgfqpoint{0.000000in}{4.800000in}}%
\pgfpathclose%
\pgfusepath{fill}%
\end{pgfscope}%
\begin{pgfscope}%
\pgfsetbuttcap%
\pgfsetmiterjoin%
\definecolor{currentfill}{rgb}{1.000000,1.000000,1.000000}%
\pgfsetfillcolor{currentfill}%
\pgfsetlinewidth{0.000000pt}%
\definecolor{currentstroke}{rgb}{0.000000,0.000000,0.000000}%
\pgfsetstrokecolor{currentstroke}%
\pgfsetstrokeopacity{0.000000}%
\pgfsetdash{}{0pt}%
\pgfpathmoveto{\pgfqpoint{0.800000in}{0.528000in}}%
\pgfpathlineto{\pgfqpoint{3.054545in}{0.528000in}}%
\pgfpathlineto{\pgfqpoint{3.054545in}{4.224000in}}%
\pgfpathlineto{\pgfqpoint{0.800000in}{4.224000in}}%
\pgfpathclose%
\pgfusepath{fill}%
\end{pgfscope}%
\begin{pgfscope}%
\pgfsetbuttcap%
\pgfsetroundjoin%
\definecolor{currentfill}{rgb}{0.000000,0.000000,0.000000}%
\pgfsetfillcolor{currentfill}%
\pgfsetlinewidth{0.803000pt}%
\definecolor{currentstroke}{rgb}{0.000000,0.000000,0.000000}%
\pgfsetstrokecolor{currentstroke}%
\pgfsetdash{}{0pt}%
\pgfsys@defobject{currentmarker}{\pgfqpoint{0.000000in}{-0.048611in}}{\pgfqpoint{0.000000in}{0.000000in}}{%
\pgfpathmoveto{\pgfqpoint{0.000000in}{0.000000in}}%
\pgfpathlineto{\pgfqpoint{0.000000in}{-0.048611in}}%
\pgfusepath{stroke,fill}%
}%
\begin{pgfscope}%
\pgfsys@transformshift{0.849926in}{0.528000in}%
\pgfsys@useobject{currentmarker}{}%
\end{pgfscope}%
\end{pgfscope}%
\begin{pgfscope}%
\definecolor{textcolor}{rgb}{0.000000,0.000000,0.000000}%
\pgfsetstrokecolor{textcolor}%
\pgfsetfillcolor{textcolor}%
\pgftext[x=0.849926in,y=0.430778in,,top]{\color{textcolor}\sffamily\fontsize{10.000000}{12.000000}\selectfont -100}%
\end{pgfscope}%
\begin{pgfscope}%
\pgfsetbuttcap%
\pgfsetroundjoin%
\definecolor{currentfill}{rgb}{0.000000,0.000000,0.000000}%
\pgfsetfillcolor{currentfill}%
\pgfsetlinewidth{0.803000pt}%
\definecolor{currentstroke}{rgb}{0.000000,0.000000,0.000000}%
\pgfsetstrokecolor{currentstroke}%
\pgfsetdash{}{0pt}%
\pgfsys@defobject{currentmarker}{\pgfqpoint{0.000000in}{-0.048611in}}{\pgfqpoint{0.000000in}{0.000000in}}{%
\pgfpathmoveto{\pgfqpoint{0.000000in}{0.000000in}}%
\pgfpathlineto{\pgfqpoint{0.000000in}{-0.048611in}}%
\pgfusepath{stroke,fill}%
}%
\begin{pgfscope}%
\pgfsys@transformshift{1.375461in}{0.528000in}%
\pgfsys@useobject{currentmarker}{}%
\end{pgfscope}%
\end{pgfscope}%
\begin{pgfscope}%
\definecolor{textcolor}{rgb}{0.000000,0.000000,0.000000}%
\pgfsetstrokecolor{textcolor}%
\pgfsetfillcolor{textcolor}%
\pgftext[x=1.375461in,y=0.430778in,,top]{\color{textcolor}\sffamily\fontsize{10.000000}{12.000000}\selectfont 0}%
\end{pgfscope}%
\begin{pgfscope}%
\pgfsetbuttcap%
\pgfsetroundjoin%
\definecolor{currentfill}{rgb}{0.000000,0.000000,0.000000}%
\pgfsetfillcolor{currentfill}%
\pgfsetlinewidth{0.803000pt}%
\definecolor{currentstroke}{rgb}{0.000000,0.000000,0.000000}%
\pgfsetstrokecolor{currentstroke}%
\pgfsetdash{}{0pt}%
\pgfsys@defobject{currentmarker}{\pgfqpoint{0.000000in}{-0.048611in}}{\pgfqpoint{0.000000in}{0.000000in}}{%
\pgfpathmoveto{\pgfqpoint{0.000000in}{0.000000in}}%
\pgfpathlineto{\pgfqpoint{0.000000in}{-0.048611in}}%
\pgfusepath{stroke,fill}%
}%
\begin{pgfscope}%
\pgfsys@transformshift{1.900996in}{0.528000in}%
\pgfsys@useobject{currentmarker}{}%
\end{pgfscope}%
\end{pgfscope}%
\begin{pgfscope}%
\definecolor{textcolor}{rgb}{0.000000,0.000000,0.000000}%
\pgfsetstrokecolor{textcolor}%
\pgfsetfillcolor{textcolor}%
\pgftext[x=1.900996in,y=0.430778in,,top]{\color{textcolor}\sffamily\fontsize{10.000000}{12.000000}\selectfont 100}%
\end{pgfscope}%
\begin{pgfscope}%
\pgfsetbuttcap%
\pgfsetroundjoin%
\definecolor{currentfill}{rgb}{0.000000,0.000000,0.000000}%
\pgfsetfillcolor{currentfill}%
\pgfsetlinewidth{0.803000pt}%
\definecolor{currentstroke}{rgb}{0.000000,0.000000,0.000000}%
\pgfsetstrokecolor{currentstroke}%
\pgfsetdash{}{0pt}%
\pgfsys@defobject{currentmarker}{\pgfqpoint{0.000000in}{-0.048611in}}{\pgfqpoint{0.000000in}{0.000000in}}{%
\pgfpathmoveto{\pgfqpoint{0.000000in}{0.000000in}}%
\pgfpathlineto{\pgfqpoint{0.000000in}{-0.048611in}}%
\pgfusepath{stroke,fill}%
}%
\begin{pgfscope}%
\pgfsys@transformshift{2.426531in}{0.528000in}%
\pgfsys@useobject{currentmarker}{}%
\end{pgfscope}%
\end{pgfscope}%
\begin{pgfscope}%
\definecolor{textcolor}{rgb}{0.000000,0.000000,0.000000}%
\pgfsetstrokecolor{textcolor}%
\pgfsetfillcolor{textcolor}%
\pgftext[x=2.426531in,y=0.430778in,,top]{\color{textcolor}\sffamily\fontsize{10.000000}{12.000000}\selectfont 200}%
\end{pgfscope}%
\begin{pgfscope}%
\pgfsetbuttcap%
\pgfsetroundjoin%
\definecolor{currentfill}{rgb}{0.000000,0.000000,0.000000}%
\pgfsetfillcolor{currentfill}%
\pgfsetlinewidth{0.803000pt}%
\definecolor{currentstroke}{rgb}{0.000000,0.000000,0.000000}%
\pgfsetstrokecolor{currentstroke}%
\pgfsetdash{}{0pt}%
\pgfsys@defobject{currentmarker}{\pgfqpoint{0.000000in}{-0.048611in}}{\pgfqpoint{0.000000in}{0.000000in}}{%
\pgfpathmoveto{\pgfqpoint{0.000000in}{0.000000in}}%
\pgfpathlineto{\pgfqpoint{0.000000in}{-0.048611in}}%
\pgfusepath{stroke,fill}%
}%
\begin{pgfscope}%
\pgfsys@transformshift{2.952066in}{0.528000in}%
\pgfsys@useobject{currentmarker}{}%
\end{pgfscope}%
\end{pgfscope}%
\begin{pgfscope}%
\definecolor{textcolor}{rgb}{0.000000,0.000000,0.000000}%
\pgfsetstrokecolor{textcolor}%
\pgfsetfillcolor{textcolor}%
\pgftext[x=2.952066in,y=0.430778in,,top]{\color{textcolor}\sffamily\fontsize{10.000000}{12.000000}\selectfont 300}%
\end{pgfscope}%
\begin{pgfscope}%
\pgfsetbuttcap%
\pgfsetroundjoin%
\definecolor{currentfill}{rgb}{0.000000,0.000000,0.000000}%
\pgfsetfillcolor{currentfill}%
\pgfsetlinewidth{0.803000pt}%
\definecolor{currentstroke}{rgb}{0.000000,0.000000,0.000000}%
\pgfsetstrokecolor{currentstroke}%
\pgfsetdash{}{0pt}%
\pgfsys@defobject{currentmarker}{\pgfqpoint{-0.048611in}{0.000000in}}{\pgfqpoint{-0.000000in}{0.000000in}}{%
\pgfpathmoveto{\pgfqpoint{-0.000000in}{0.000000in}}%
\pgfpathlineto{\pgfqpoint{-0.048611in}{0.000000in}}%
\pgfusepath{stroke,fill}%
}%
\begin{pgfscope}%
\pgfsys@transformshift{0.800000in}{0.696000in}%
\pgfsys@useobject{currentmarker}{}%
\end{pgfscope}%
\end{pgfscope}%
\begin{pgfscope}%
\definecolor{textcolor}{rgb}{0.000000,0.000000,0.000000}%
\pgfsetstrokecolor{textcolor}%
\pgfsetfillcolor{textcolor}%
\pgftext[x=0.481898in, y=0.643238in, left, base]{\color{textcolor}\sffamily\fontsize{10.000000}{12.000000}\selectfont 6.0}%
\end{pgfscope}%
\begin{pgfscope}%
\pgfsetbuttcap%
\pgfsetroundjoin%
\definecolor{currentfill}{rgb}{0.000000,0.000000,0.000000}%
\pgfsetfillcolor{currentfill}%
\pgfsetlinewidth{0.803000pt}%
\definecolor{currentstroke}{rgb}{0.000000,0.000000,0.000000}%
\pgfsetstrokecolor{currentstroke}%
\pgfsetdash{}{0pt}%
\pgfsys@defobject{currentmarker}{\pgfqpoint{-0.048611in}{0.000000in}}{\pgfqpoint{-0.000000in}{0.000000in}}{%
\pgfpathmoveto{\pgfqpoint{-0.000000in}{0.000000in}}%
\pgfpathlineto{\pgfqpoint{-0.048611in}{0.000000in}}%
\pgfusepath{stroke,fill}%
}%
\begin{pgfscope}%
\pgfsys@transformshift{0.800000in}{1.368412in}%
\pgfsys@useobject{currentmarker}{}%
\end{pgfscope}%
\end{pgfscope}%
\begin{pgfscope}%
\definecolor{textcolor}{rgb}{0.000000,0.000000,0.000000}%
\pgfsetstrokecolor{textcolor}%
\pgfsetfillcolor{textcolor}%
\pgftext[x=0.481898in, y=1.315650in, left, base]{\color{textcolor}\sffamily\fontsize{10.000000}{12.000000}\selectfont 6.2}%
\end{pgfscope}%
\begin{pgfscope}%
\pgfsetbuttcap%
\pgfsetroundjoin%
\definecolor{currentfill}{rgb}{0.000000,0.000000,0.000000}%
\pgfsetfillcolor{currentfill}%
\pgfsetlinewidth{0.803000pt}%
\definecolor{currentstroke}{rgb}{0.000000,0.000000,0.000000}%
\pgfsetstrokecolor{currentstroke}%
\pgfsetdash{}{0pt}%
\pgfsys@defobject{currentmarker}{\pgfqpoint{-0.048611in}{0.000000in}}{\pgfqpoint{-0.000000in}{0.000000in}}{%
\pgfpathmoveto{\pgfqpoint{-0.000000in}{0.000000in}}%
\pgfpathlineto{\pgfqpoint{-0.048611in}{0.000000in}}%
\pgfusepath{stroke,fill}%
}%
\begin{pgfscope}%
\pgfsys@transformshift{0.800000in}{2.040823in}%
\pgfsys@useobject{currentmarker}{}%
\end{pgfscope}%
\end{pgfscope}%
\begin{pgfscope}%
\definecolor{textcolor}{rgb}{0.000000,0.000000,0.000000}%
\pgfsetstrokecolor{textcolor}%
\pgfsetfillcolor{textcolor}%
\pgftext[x=0.481898in, y=1.988062in, left, base]{\color{textcolor}\sffamily\fontsize{10.000000}{12.000000}\selectfont 6.4}%
\end{pgfscope}%
\begin{pgfscope}%
\pgfsetbuttcap%
\pgfsetroundjoin%
\definecolor{currentfill}{rgb}{0.000000,0.000000,0.000000}%
\pgfsetfillcolor{currentfill}%
\pgfsetlinewidth{0.803000pt}%
\definecolor{currentstroke}{rgb}{0.000000,0.000000,0.000000}%
\pgfsetstrokecolor{currentstroke}%
\pgfsetdash{}{0pt}%
\pgfsys@defobject{currentmarker}{\pgfqpoint{-0.048611in}{0.000000in}}{\pgfqpoint{-0.000000in}{0.000000in}}{%
\pgfpathmoveto{\pgfqpoint{-0.000000in}{0.000000in}}%
\pgfpathlineto{\pgfqpoint{-0.048611in}{0.000000in}}%
\pgfusepath{stroke,fill}%
}%
\begin{pgfscope}%
\pgfsys@transformshift{0.800000in}{2.713235in}%
\pgfsys@useobject{currentmarker}{}%
\end{pgfscope}%
\end{pgfscope}%
\begin{pgfscope}%
\definecolor{textcolor}{rgb}{0.000000,0.000000,0.000000}%
\pgfsetstrokecolor{textcolor}%
\pgfsetfillcolor{textcolor}%
\pgftext[x=0.481898in, y=2.660473in, left, base]{\color{textcolor}\sffamily\fontsize{10.000000}{12.000000}\selectfont 6.6}%
\end{pgfscope}%
\begin{pgfscope}%
\pgfsetbuttcap%
\pgfsetroundjoin%
\definecolor{currentfill}{rgb}{0.000000,0.000000,0.000000}%
\pgfsetfillcolor{currentfill}%
\pgfsetlinewidth{0.803000pt}%
\definecolor{currentstroke}{rgb}{0.000000,0.000000,0.000000}%
\pgfsetstrokecolor{currentstroke}%
\pgfsetdash{}{0pt}%
\pgfsys@defobject{currentmarker}{\pgfqpoint{-0.048611in}{0.000000in}}{\pgfqpoint{-0.000000in}{0.000000in}}{%
\pgfpathmoveto{\pgfqpoint{-0.000000in}{0.000000in}}%
\pgfpathlineto{\pgfqpoint{-0.048611in}{0.000000in}}%
\pgfusepath{stroke,fill}%
}%
\begin{pgfscope}%
\pgfsys@transformshift{0.800000in}{3.385646in}%
\pgfsys@useobject{currentmarker}{}%
\end{pgfscope}%
\end{pgfscope}%
\begin{pgfscope}%
\definecolor{textcolor}{rgb}{0.000000,0.000000,0.000000}%
\pgfsetstrokecolor{textcolor}%
\pgfsetfillcolor{textcolor}%
\pgftext[x=0.481898in, y=3.332885in, left, base]{\color{textcolor}\sffamily\fontsize{10.000000}{12.000000}\selectfont 6.8}%
\end{pgfscope}%
\begin{pgfscope}%
\pgfsetbuttcap%
\pgfsetroundjoin%
\definecolor{currentfill}{rgb}{0.000000,0.000000,0.000000}%
\pgfsetfillcolor{currentfill}%
\pgfsetlinewidth{0.803000pt}%
\definecolor{currentstroke}{rgb}{0.000000,0.000000,0.000000}%
\pgfsetstrokecolor{currentstroke}%
\pgfsetdash{}{0pt}%
\pgfsys@defobject{currentmarker}{\pgfqpoint{-0.048611in}{0.000000in}}{\pgfqpoint{-0.000000in}{0.000000in}}{%
\pgfpathmoveto{\pgfqpoint{-0.000000in}{0.000000in}}%
\pgfpathlineto{\pgfqpoint{-0.048611in}{0.000000in}}%
\pgfusepath{stroke,fill}%
}%
\begin{pgfscope}%
\pgfsys@transformshift{0.800000in}{4.058058in}%
\pgfsys@useobject{currentmarker}{}%
\end{pgfscope}%
\end{pgfscope}%
\begin{pgfscope}%
\definecolor{textcolor}{rgb}{0.000000,0.000000,0.000000}%
\pgfsetstrokecolor{textcolor}%
\pgfsetfillcolor{textcolor}%
\pgftext[x=0.481898in, y=4.005296in, left, base]{\color{textcolor}\sffamily\fontsize{10.000000}{12.000000}\selectfont 7.0}%
\end{pgfscope}%
\begin{pgfscope}%
\pgfpathrectangle{\pgfqpoint{0.800000in}{0.528000in}}{\pgfqpoint{2.254545in}{3.696000in}}%
\pgfusepath{clip}%
\pgfsetrectcap%
\pgfsetroundjoin%
\pgfsetlinewidth{1.505625pt}%
\definecolor{currentstroke}{rgb}{0.121569,0.466667,0.705882}%
\pgfsetstrokecolor{currentstroke}%
\pgfsetdash{}{0pt}%
\pgfpathmoveto{\pgfqpoint{0.902479in}{0.696000in}}%
\pgfpathlineto{\pgfqpoint{1.374356in}{0.696000in}}%
\pgfpathlineto{\pgfqpoint{1.376408in}{4.056000in}}%
\pgfpathlineto{\pgfqpoint{1.413337in}{3.978571in}}%
\pgfpathlineto{\pgfqpoint{1.450267in}{3.906396in}}%
\pgfpathlineto{\pgfqpoint{1.487196in}{3.839119in}}%
\pgfpathlineto{\pgfqpoint{1.524126in}{3.776407in}}%
\pgfpathlineto{\pgfqpoint{1.561055in}{3.717951in}}%
\pgfpathlineto{\pgfqpoint{1.597985in}{3.663461in}}%
\pgfpathlineto{\pgfqpoint{1.636966in}{3.609951in}}%
\pgfpathlineto{\pgfqpoint{1.675947in}{3.560266in}}%
\pgfpathlineto{\pgfqpoint{1.714928in}{3.514133in}}%
\pgfpathlineto{\pgfqpoint{1.753909in}{3.471298in}}%
\pgfpathlineto{\pgfqpoint{1.794942in}{3.429512in}}%
\pgfpathlineto{\pgfqpoint{1.835975in}{3.390865in}}%
\pgfpathlineto{\pgfqpoint{1.879059in}{3.353406in}}%
\pgfpathlineto{\pgfqpoint{1.922144in}{3.318895in}}%
\pgfpathlineto{\pgfqpoint{1.967280in}{3.285651in}}%
\pgfpathlineto{\pgfqpoint{2.012416in}{3.255142in}}%
\pgfpathlineto{\pgfqpoint{2.059603in}{3.225928in}}%
\pgfpathlineto{\pgfqpoint{2.108843in}{3.198116in}}%
\pgfpathlineto{\pgfqpoint{2.160134in}{3.171786in}}%
\pgfpathlineto{\pgfqpoint{2.213476in}{3.146997in}}%
\pgfpathlineto{\pgfqpoint{2.268871in}{3.123784in}}%
\pgfpathlineto{\pgfqpoint{2.326316in}{3.102160in}}%
\pgfpathlineto{\pgfqpoint{2.385814in}{3.082122in}}%
\pgfpathlineto{\pgfqpoint{2.449415in}{3.063068in}}%
\pgfpathlineto{\pgfqpoint{2.517119in}{3.045165in}}%
\pgfpathlineto{\pgfqpoint{2.588926in}{3.028536in}}%
\pgfpathlineto{\pgfqpoint{2.664837in}{3.013260in}}%
\pgfpathlineto{\pgfqpoint{2.746902in}{2.999048in}}%
\pgfpathlineto{\pgfqpoint{2.835123in}{2.986053in}}%
\pgfpathlineto{\pgfqpoint{2.931550in}{2.974135in}}%
\pgfpathlineto{\pgfqpoint{2.952066in}{2.971870in}}%
\pgfpathlineto{\pgfqpoint{2.952066in}{2.971870in}}%
\pgfusepath{stroke}%
\end{pgfscope}%
\begin{pgfscope}%
\pgfsetrectcap%
\pgfsetmiterjoin%
\pgfsetlinewidth{0.803000pt}%
\definecolor{currentstroke}{rgb}{0.000000,0.000000,0.000000}%
\pgfsetstrokecolor{currentstroke}%
\pgfsetdash{}{0pt}%
\pgfpathmoveto{\pgfqpoint{0.800000in}{0.528000in}}%
\pgfpathlineto{\pgfqpoint{0.800000in}{4.224000in}}%
\pgfusepath{stroke}%
\end{pgfscope}%
\begin{pgfscope}%
\pgfsetrectcap%
\pgfsetmiterjoin%
\pgfsetlinewidth{0.803000pt}%
\definecolor{currentstroke}{rgb}{0.000000,0.000000,0.000000}%
\pgfsetstrokecolor{currentstroke}%
\pgfsetdash{}{0pt}%
\pgfpathmoveto{\pgfqpoint{3.054545in}{0.528000in}}%
\pgfpathlineto{\pgfqpoint{3.054545in}{4.224000in}}%
\pgfusepath{stroke}%
\end{pgfscope}%
\begin{pgfscope}%
\pgfsetrectcap%
\pgfsetmiterjoin%
\pgfsetlinewidth{0.803000pt}%
\definecolor{currentstroke}{rgb}{0.000000,0.000000,0.000000}%
\pgfsetstrokecolor{currentstroke}%
\pgfsetdash{}{0pt}%
\pgfpathmoveto{\pgfqpoint{0.800000in}{0.528000in}}%
\pgfpathlineto{\pgfqpoint{3.054545in}{0.528000in}}%
\pgfusepath{stroke}%
\end{pgfscope}%
\begin{pgfscope}%
\pgfsetrectcap%
\pgfsetmiterjoin%
\pgfsetlinewidth{0.803000pt}%
\definecolor{currentstroke}{rgb}{0.000000,0.000000,0.000000}%
\pgfsetstrokecolor{currentstroke}%
\pgfsetdash{}{0pt}%
\pgfpathmoveto{\pgfqpoint{0.800000in}{4.224000in}}%
\pgfpathlineto{\pgfqpoint{3.054545in}{4.224000in}}%
\pgfusepath{stroke}%
\end{pgfscope}%
\begin{pgfscope}%
\definecolor{textcolor}{rgb}{0.000000,0.000000,0.000000}%
\pgfsetstrokecolor{textcolor}%
\pgfsetfillcolor{textcolor}%
\pgftext[x=1.927273in,y=4.307333in,,base]{\color{textcolor}\sffamily\fontsize{12.000000}{14.400000}\selectfont \(\displaystyle v_a(t)\)}%
\end{pgfscope}%
\begin{pgfscope}%
\pgfsetbuttcap%
\pgfsetmiterjoin%
\definecolor{currentfill}{rgb}{1.000000,1.000000,1.000000}%
\pgfsetfillcolor{currentfill}%
\pgfsetlinewidth{0.000000pt}%
\definecolor{currentstroke}{rgb}{0.000000,0.000000,0.000000}%
\pgfsetstrokecolor{currentstroke}%
\pgfsetstrokeopacity{0.000000}%
\pgfsetdash{}{0pt}%
\pgfpathmoveto{\pgfqpoint{3.505455in}{0.528000in}}%
\pgfpathlineto{\pgfqpoint{5.760000in}{0.528000in}}%
\pgfpathlineto{\pgfqpoint{5.760000in}{4.224000in}}%
\pgfpathlineto{\pgfqpoint{3.505455in}{4.224000in}}%
\pgfpathclose%
\pgfusepath{fill}%
\end{pgfscope}%
\begin{pgfscope}%
\pgfsetbuttcap%
\pgfsetroundjoin%
\definecolor{currentfill}{rgb}{0.000000,0.000000,0.000000}%
\pgfsetfillcolor{currentfill}%
\pgfsetlinewidth{0.803000pt}%
\definecolor{currentstroke}{rgb}{0.000000,0.000000,0.000000}%
\pgfsetstrokecolor{currentstroke}%
\pgfsetdash{}{0pt}%
\pgfsys@defobject{currentmarker}{\pgfqpoint{0.000000in}{-0.048611in}}{\pgfqpoint{0.000000in}{0.000000in}}{%
\pgfpathmoveto{\pgfqpoint{0.000000in}{0.000000in}}%
\pgfpathlineto{\pgfqpoint{0.000000in}{-0.048611in}}%
\pgfusepath{stroke,fill}%
}%
\begin{pgfscope}%
\pgfsys@transformshift{3.555380in}{0.528000in}%
\pgfsys@useobject{currentmarker}{}%
\end{pgfscope}%
\end{pgfscope}%
\begin{pgfscope}%
\definecolor{textcolor}{rgb}{0.000000,0.000000,0.000000}%
\pgfsetstrokecolor{textcolor}%
\pgfsetfillcolor{textcolor}%
\pgftext[x=3.555380in,y=0.430778in,,top]{\color{textcolor}\sffamily\fontsize{10.000000}{12.000000}\selectfont -100}%
\end{pgfscope}%
\begin{pgfscope}%
\pgfsetbuttcap%
\pgfsetroundjoin%
\definecolor{currentfill}{rgb}{0.000000,0.000000,0.000000}%
\pgfsetfillcolor{currentfill}%
\pgfsetlinewidth{0.803000pt}%
\definecolor{currentstroke}{rgb}{0.000000,0.000000,0.000000}%
\pgfsetstrokecolor{currentstroke}%
\pgfsetdash{}{0pt}%
\pgfsys@defobject{currentmarker}{\pgfqpoint{0.000000in}{-0.048611in}}{\pgfqpoint{0.000000in}{0.000000in}}{%
\pgfpathmoveto{\pgfqpoint{0.000000in}{0.000000in}}%
\pgfpathlineto{\pgfqpoint{0.000000in}{-0.048611in}}%
\pgfusepath{stroke,fill}%
}%
\begin{pgfscope}%
\pgfsys@transformshift{4.080915in}{0.528000in}%
\pgfsys@useobject{currentmarker}{}%
\end{pgfscope}%
\end{pgfscope}%
\begin{pgfscope}%
\definecolor{textcolor}{rgb}{0.000000,0.000000,0.000000}%
\pgfsetstrokecolor{textcolor}%
\pgfsetfillcolor{textcolor}%
\pgftext[x=4.080915in,y=0.430778in,,top]{\color{textcolor}\sffamily\fontsize{10.000000}{12.000000}\selectfont 0}%
\end{pgfscope}%
\begin{pgfscope}%
\pgfsetbuttcap%
\pgfsetroundjoin%
\definecolor{currentfill}{rgb}{0.000000,0.000000,0.000000}%
\pgfsetfillcolor{currentfill}%
\pgfsetlinewidth{0.803000pt}%
\definecolor{currentstroke}{rgb}{0.000000,0.000000,0.000000}%
\pgfsetstrokecolor{currentstroke}%
\pgfsetdash{}{0pt}%
\pgfsys@defobject{currentmarker}{\pgfqpoint{0.000000in}{-0.048611in}}{\pgfqpoint{0.000000in}{0.000000in}}{%
\pgfpathmoveto{\pgfqpoint{0.000000in}{0.000000in}}%
\pgfpathlineto{\pgfqpoint{0.000000in}{-0.048611in}}%
\pgfusepath{stroke,fill}%
}%
\begin{pgfscope}%
\pgfsys@transformshift{4.606451in}{0.528000in}%
\pgfsys@useobject{currentmarker}{}%
\end{pgfscope}%
\end{pgfscope}%
\begin{pgfscope}%
\definecolor{textcolor}{rgb}{0.000000,0.000000,0.000000}%
\pgfsetstrokecolor{textcolor}%
\pgfsetfillcolor{textcolor}%
\pgftext[x=4.606451in,y=0.430778in,,top]{\color{textcolor}\sffamily\fontsize{10.000000}{12.000000}\selectfont 100}%
\end{pgfscope}%
\begin{pgfscope}%
\pgfsetbuttcap%
\pgfsetroundjoin%
\definecolor{currentfill}{rgb}{0.000000,0.000000,0.000000}%
\pgfsetfillcolor{currentfill}%
\pgfsetlinewidth{0.803000pt}%
\definecolor{currentstroke}{rgb}{0.000000,0.000000,0.000000}%
\pgfsetstrokecolor{currentstroke}%
\pgfsetdash{}{0pt}%
\pgfsys@defobject{currentmarker}{\pgfqpoint{0.000000in}{-0.048611in}}{\pgfqpoint{0.000000in}{0.000000in}}{%
\pgfpathmoveto{\pgfqpoint{0.000000in}{0.000000in}}%
\pgfpathlineto{\pgfqpoint{0.000000in}{-0.048611in}}%
\pgfusepath{stroke,fill}%
}%
\begin{pgfscope}%
\pgfsys@transformshift{5.131986in}{0.528000in}%
\pgfsys@useobject{currentmarker}{}%
\end{pgfscope}%
\end{pgfscope}%
\begin{pgfscope}%
\definecolor{textcolor}{rgb}{0.000000,0.000000,0.000000}%
\pgfsetstrokecolor{textcolor}%
\pgfsetfillcolor{textcolor}%
\pgftext[x=5.131986in,y=0.430778in,,top]{\color{textcolor}\sffamily\fontsize{10.000000}{12.000000}\selectfont 200}%
\end{pgfscope}%
\begin{pgfscope}%
\pgfsetbuttcap%
\pgfsetroundjoin%
\definecolor{currentfill}{rgb}{0.000000,0.000000,0.000000}%
\pgfsetfillcolor{currentfill}%
\pgfsetlinewidth{0.803000pt}%
\definecolor{currentstroke}{rgb}{0.000000,0.000000,0.000000}%
\pgfsetstrokecolor{currentstroke}%
\pgfsetdash{}{0pt}%
\pgfsys@defobject{currentmarker}{\pgfqpoint{0.000000in}{-0.048611in}}{\pgfqpoint{0.000000in}{0.000000in}}{%
\pgfpathmoveto{\pgfqpoint{0.000000in}{0.000000in}}%
\pgfpathlineto{\pgfqpoint{0.000000in}{-0.048611in}}%
\pgfusepath{stroke,fill}%
}%
\begin{pgfscope}%
\pgfsys@transformshift{5.657521in}{0.528000in}%
\pgfsys@useobject{currentmarker}{}%
\end{pgfscope}%
\end{pgfscope}%
\begin{pgfscope}%
\definecolor{textcolor}{rgb}{0.000000,0.000000,0.000000}%
\pgfsetstrokecolor{textcolor}%
\pgfsetfillcolor{textcolor}%
\pgftext[x=5.657521in,y=0.430778in,,top]{\color{textcolor}\sffamily\fontsize{10.000000}{12.000000}\selectfont 300}%
\end{pgfscope}%
\begin{pgfscope}%
\pgfsetbuttcap%
\pgfsetroundjoin%
\definecolor{currentfill}{rgb}{0.000000,0.000000,0.000000}%
\pgfsetfillcolor{currentfill}%
\pgfsetlinewidth{0.803000pt}%
\definecolor{currentstroke}{rgb}{0.000000,0.000000,0.000000}%
\pgfsetstrokecolor{currentstroke}%
\pgfsetdash{}{0pt}%
\pgfsys@defobject{currentmarker}{\pgfqpoint{-0.048611in}{0.000000in}}{\pgfqpoint{-0.000000in}{0.000000in}}{%
\pgfpathmoveto{\pgfqpoint{-0.000000in}{0.000000in}}%
\pgfpathlineto{\pgfqpoint{-0.048611in}{0.000000in}}%
\pgfusepath{stroke,fill}%
}%
\begin{pgfscope}%
\pgfsys@transformshift{3.505455in}{0.891863in}%
\pgfsys@useobject{currentmarker}{}%
\end{pgfscope}%
\end{pgfscope}%
\begin{pgfscope}%
\definecolor{textcolor}{rgb}{0.000000,0.000000,0.000000}%
\pgfsetstrokecolor{textcolor}%
\pgfsetfillcolor{textcolor}%
\pgftext[x=3.187353in, y=0.839101in, left, base]{\color{textcolor}\sffamily\fontsize{10.000000}{12.000000}\selectfont 3.4}%
\end{pgfscope}%
\begin{pgfscope}%
\pgfsetbuttcap%
\pgfsetroundjoin%
\definecolor{currentfill}{rgb}{0.000000,0.000000,0.000000}%
\pgfsetfillcolor{currentfill}%
\pgfsetlinewidth{0.803000pt}%
\definecolor{currentstroke}{rgb}{0.000000,0.000000,0.000000}%
\pgfsetstrokecolor{currentstroke}%
\pgfsetdash{}{0pt}%
\pgfsys@defobject{currentmarker}{\pgfqpoint{-0.048611in}{0.000000in}}{\pgfqpoint{-0.000000in}{0.000000in}}{%
\pgfpathmoveto{\pgfqpoint{-0.000000in}{0.000000in}}%
\pgfpathlineto{\pgfqpoint{-0.048611in}{0.000000in}}%
\pgfusepath{stroke,fill}%
}%
\begin{pgfscope}%
\pgfsys@transformshift{3.505455in}{1.419219in}%
\pgfsys@useobject{currentmarker}{}%
\end{pgfscope}%
\end{pgfscope}%
\begin{pgfscope}%
\definecolor{textcolor}{rgb}{0.000000,0.000000,0.000000}%
\pgfsetstrokecolor{textcolor}%
\pgfsetfillcolor{textcolor}%
\pgftext[x=3.187353in, y=1.366458in, left, base]{\color{textcolor}\sffamily\fontsize{10.000000}{12.000000}\selectfont 3.5}%
\end{pgfscope}%
\begin{pgfscope}%
\pgfsetbuttcap%
\pgfsetroundjoin%
\definecolor{currentfill}{rgb}{0.000000,0.000000,0.000000}%
\pgfsetfillcolor{currentfill}%
\pgfsetlinewidth{0.803000pt}%
\definecolor{currentstroke}{rgb}{0.000000,0.000000,0.000000}%
\pgfsetstrokecolor{currentstroke}%
\pgfsetdash{}{0pt}%
\pgfsys@defobject{currentmarker}{\pgfqpoint{-0.048611in}{0.000000in}}{\pgfqpoint{-0.000000in}{0.000000in}}{%
\pgfpathmoveto{\pgfqpoint{-0.000000in}{0.000000in}}%
\pgfpathlineto{\pgfqpoint{-0.048611in}{0.000000in}}%
\pgfusepath{stroke,fill}%
}%
\begin{pgfscope}%
\pgfsys@transformshift{3.505455in}{1.946575in}%
\pgfsys@useobject{currentmarker}{}%
\end{pgfscope}%
\end{pgfscope}%
\begin{pgfscope}%
\definecolor{textcolor}{rgb}{0.000000,0.000000,0.000000}%
\pgfsetstrokecolor{textcolor}%
\pgfsetfillcolor{textcolor}%
\pgftext[x=3.187353in, y=1.893814in, left, base]{\color{textcolor}\sffamily\fontsize{10.000000}{12.000000}\selectfont 3.6}%
\end{pgfscope}%
\begin{pgfscope}%
\pgfsetbuttcap%
\pgfsetroundjoin%
\definecolor{currentfill}{rgb}{0.000000,0.000000,0.000000}%
\pgfsetfillcolor{currentfill}%
\pgfsetlinewidth{0.803000pt}%
\definecolor{currentstroke}{rgb}{0.000000,0.000000,0.000000}%
\pgfsetstrokecolor{currentstroke}%
\pgfsetdash{}{0pt}%
\pgfsys@defobject{currentmarker}{\pgfqpoint{-0.048611in}{0.000000in}}{\pgfqpoint{-0.000000in}{0.000000in}}{%
\pgfpathmoveto{\pgfqpoint{-0.000000in}{0.000000in}}%
\pgfpathlineto{\pgfqpoint{-0.048611in}{0.000000in}}%
\pgfusepath{stroke,fill}%
}%
\begin{pgfscope}%
\pgfsys@transformshift{3.505455in}{2.473931in}%
\pgfsys@useobject{currentmarker}{}%
\end{pgfscope}%
\end{pgfscope}%
\begin{pgfscope}%
\definecolor{textcolor}{rgb}{0.000000,0.000000,0.000000}%
\pgfsetstrokecolor{textcolor}%
\pgfsetfillcolor{textcolor}%
\pgftext[x=3.187353in, y=2.421170in, left, base]{\color{textcolor}\sffamily\fontsize{10.000000}{12.000000}\selectfont 3.7}%
\end{pgfscope}%
\begin{pgfscope}%
\pgfsetbuttcap%
\pgfsetroundjoin%
\definecolor{currentfill}{rgb}{0.000000,0.000000,0.000000}%
\pgfsetfillcolor{currentfill}%
\pgfsetlinewidth{0.803000pt}%
\definecolor{currentstroke}{rgb}{0.000000,0.000000,0.000000}%
\pgfsetstrokecolor{currentstroke}%
\pgfsetdash{}{0pt}%
\pgfsys@defobject{currentmarker}{\pgfqpoint{-0.048611in}{0.000000in}}{\pgfqpoint{-0.000000in}{0.000000in}}{%
\pgfpathmoveto{\pgfqpoint{-0.000000in}{0.000000in}}%
\pgfpathlineto{\pgfqpoint{-0.048611in}{0.000000in}}%
\pgfusepath{stroke,fill}%
}%
\begin{pgfscope}%
\pgfsys@transformshift{3.505455in}{3.001288in}%
\pgfsys@useobject{currentmarker}{}%
\end{pgfscope}%
\end{pgfscope}%
\begin{pgfscope}%
\definecolor{textcolor}{rgb}{0.000000,0.000000,0.000000}%
\pgfsetstrokecolor{textcolor}%
\pgfsetfillcolor{textcolor}%
\pgftext[x=3.187353in, y=2.948526in, left, base]{\color{textcolor}\sffamily\fontsize{10.000000}{12.000000}\selectfont 3.8}%
\end{pgfscope}%
\begin{pgfscope}%
\pgfsetbuttcap%
\pgfsetroundjoin%
\definecolor{currentfill}{rgb}{0.000000,0.000000,0.000000}%
\pgfsetfillcolor{currentfill}%
\pgfsetlinewidth{0.803000pt}%
\definecolor{currentstroke}{rgb}{0.000000,0.000000,0.000000}%
\pgfsetstrokecolor{currentstroke}%
\pgfsetdash{}{0pt}%
\pgfsys@defobject{currentmarker}{\pgfqpoint{-0.048611in}{0.000000in}}{\pgfqpoint{-0.000000in}{0.000000in}}{%
\pgfpathmoveto{\pgfqpoint{-0.000000in}{0.000000in}}%
\pgfpathlineto{\pgfqpoint{-0.048611in}{0.000000in}}%
\pgfusepath{stroke,fill}%
}%
\begin{pgfscope}%
\pgfsys@transformshift{3.505455in}{3.528644in}%
\pgfsys@useobject{currentmarker}{}%
\end{pgfscope}%
\end{pgfscope}%
\begin{pgfscope}%
\definecolor{textcolor}{rgb}{0.000000,0.000000,0.000000}%
\pgfsetstrokecolor{textcolor}%
\pgfsetfillcolor{textcolor}%
\pgftext[x=3.187353in, y=3.475882in, left, base]{\color{textcolor}\sffamily\fontsize{10.000000}{12.000000}\selectfont 3.9}%
\end{pgfscope}%
\begin{pgfscope}%
\pgfsetbuttcap%
\pgfsetroundjoin%
\definecolor{currentfill}{rgb}{0.000000,0.000000,0.000000}%
\pgfsetfillcolor{currentfill}%
\pgfsetlinewidth{0.803000pt}%
\definecolor{currentstroke}{rgb}{0.000000,0.000000,0.000000}%
\pgfsetstrokecolor{currentstroke}%
\pgfsetdash{}{0pt}%
\pgfsys@defobject{currentmarker}{\pgfqpoint{-0.048611in}{0.000000in}}{\pgfqpoint{-0.000000in}{0.000000in}}{%
\pgfpathmoveto{\pgfqpoint{-0.000000in}{0.000000in}}%
\pgfpathlineto{\pgfqpoint{-0.048611in}{0.000000in}}%
\pgfusepath{stroke,fill}%
}%
\begin{pgfscope}%
\pgfsys@transformshift{3.505455in}{4.056000in}%
\pgfsys@useobject{currentmarker}{}%
\end{pgfscope}%
\end{pgfscope}%
\begin{pgfscope}%
\definecolor{textcolor}{rgb}{0.000000,0.000000,0.000000}%
\pgfsetstrokecolor{textcolor}%
\pgfsetfillcolor{textcolor}%
\pgftext[x=3.187353in, y=4.003238in, left, base]{\color{textcolor}\sffamily\fontsize{10.000000}{12.000000}\selectfont 4.0}%
\end{pgfscope}%
\begin{pgfscope}%
\pgfpathrectangle{\pgfqpoint{3.505455in}{0.528000in}}{\pgfqpoint{2.254545in}{3.696000in}}%
\pgfusepath{clip}%
\pgfsetrectcap%
\pgfsetroundjoin%
\pgfsetlinewidth{1.505625pt}%
\definecolor{currentstroke}{rgb}{0.121569,0.466667,0.705882}%
\pgfsetstrokecolor{currentstroke}%
\pgfsetdash{}{0pt}%
\pgfpathmoveto{\pgfqpoint{3.607934in}{4.056000in}}%
\pgfpathlineto{\pgfqpoint{4.079811in}{4.056000in}}%
\pgfpathlineto{\pgfqpoint{4.081862in}{3.996999in}}%
\pgfpathlineto{\pgfqpoint{4.116740in}{3.773908in}}%
\pgfpathlineto{\pgfqpoint{4.151618in}{3.565142in}}%
\pgfpathlineto{\pgfqpoint{4.186496in}{3.369781in}}%
\pgfpathlineto{\pgfqpoint{4.221374in}{3.186965in}}%
\pgfpathlineto{\pgfqpoint{4.256252in}{3.015888in}}%
\pgfpathlineto{\pgfqpoint{4.291129in}{2.855797in}}%
\pgfpathlineto{\pgfqpoint{4.326007in}{2.705985in}}%
\pgfpathlineto{\pgfqpoint{4.360885in}{2.565792in}}%
\pgfpathlineto{\pgfqpoint{4.395763in}{2.434602in}}%
\pgfpathlineto{\pgfqpoint{4.430641in}{2.311836in}}%
\pgfpathlineto{\pgfqpoint{4.465519in}{2.196953in}}%
\pgfpathlineto{\pgfqpoint{4.500397in}{2.089446in}}%
\pgfpathlineto{\pgfqpoint{4.535274in}{1.988843in}}%
\pgfpathlineto{\pgfqpoint{4.570152in}{1.894700in}}%
\pgfpathlineto{\pgfqpoint{4.605030in}{1.806602in}}%
\pgfpathlineto{\pgfqpoint{4.641960in}{1.719480in}}%
\pgfpathlineto{\pgfqpoint{4.678889in}{1.638269in}}%
\pgfpathlineto{\pgfqpoint{4.715819in}{1.562570in}}%
\pgfpathlineto{\pgfqpoint{4.752748in}{1.492007in}}%
\pgfpathlineto{\pgfqpoint{4.789678in}{1.426233in}}%
\pgfpathlineto{\pgfqpoint{4.826607in}{1.364921in}}%
\pgfpathlineto{\pgfqpoint{4.863537in}{1.307771in}}%
\pgfpathlineto{\pgfqpoint{4.902518in}{1.251647in}}%
\pgfpathlineto{\pgfqpoint{4.941499in}{1.199536in}}%
\pgfpathlineto{\pgfqpoint{4.980480in}{1.151150in}}%
\pgfpathlineto{\pgfqpoint{5.019461in}{1.106223in}}%
\pgfpathlineto{\pgfqpoint{5.060494in}{1.062397in}}%
\pgfpathlineto{\pgfqpoint{5.101527in}{1.021862in}}%
\pgfpathlineto{\pgfqpoint{5.142559in}{0.984372in}}%
\pgfpathlineto{\pgfqpoint{5.185644in}{0.948034in}}%
\pgfpathlineto{\pgfqpoint{5.228728in}{0.914557in}}%
\pgfpathlineto{\pgfqpoint{5.273864in}{0.882308in}}%
\pgfpathlineto{\pgfqpoint{5.321052in}{0.851427in}}%
\pgfpathlineto{\pgfqpoint{5.368240in}{0.823198in}}%
\pgfpathlineto{\pgfqpoint{5.417479in}{0.796323in}}%
\pgfpathlineto{\pgfqpoint{5.468770in}{0.770881in}}%
\pgfpathlineto{\pgfqpoint{5.522113in}{0.746928in}}%
\pgfpathlineto{\pgfqpoint{5.577507in}{0.724498in}}%
\pgfpathlineto{\pgfqpoint{5.634953in}{0.703604in}}%
\pgfpathlineto{\pgfqpoint{5.657521in}{0.696000in}}%
\pgfpathlineto{\pgfqpoint{5.657521in}{0.696000in}}%
\pgfusepath{stroke}%
\end{pgfscope}%
\begin{pgfscope}%
\pgfsetrectcap%
\pgfsetmiterjoin%
\pgfsetlinewidth{0.803000pt}%
\definecolor{currentstroke}{rgb}{0.000000,0.000000,0.000000}%
\pgfsetstrokecolor{currentstroke}%
\pgfsetdash{}{0pt}%
\pgfpathmoveto{\pgfqpoint{3.505455in}{0.528000in}}%
\pgfpathlineto{\pgfqpoint{3.505455in}{4.224000in}}%
\pgfusepath{stroke}%
\end{pgfscope}%
\begin{pgfscope}%
\pgfsetrectcap%
\pgfsetmiterjoin%
\pgfsetlinewidth{0.803000pt}%
\definecolor{currentstroke}{rgb}{0.000000,0.000000,0.000000}%
\pgfsetstrokecolor{currentstroke}%
\pgfsetdash{}{0pt}%
\pgfpathmoveto{\pgfqpoint{5.760000in}{0.528000in}}%
\pgfpathlineto{\pgfqpoint{5.760000in}{4.224000in}}%
\pgfusepath{stroke}%
\end{pgfscope}%
\begin{pgfscope}%
\pgfsetrectcap%
\pgfsetmiterjoin%
\pgfsetlinewidth{0.803000pt}%
\definecolor{currentstroke}{rgb}{0.000000,0.000000,0.000000}%
\pgfsetstrokecolor{currentstroke}%
\pgfsetdash{}{0pt}%
\pgfpathmoveto{\pgfqpoint{3.505455in}{0.528000in}}%
\pgfpathlineto{\pgfqpoint{5.760000in}{0.528000in}}%
\pgfusepath{stroke}%
\end{pgfscope}%
\begin{pgfscope}%
\pgfsetrectcap%
\pgfsetmiterjoin%
\pgfsetlinewidth{0.803000pt}%
\definecolor{currentstroke}{rgb}{0.000000,0.000000,0.000000}%
\pgfsetstrokecolor{currentstroke}%
\pgfsetdash{}{0pt}%
\pgfpathmoveto{\pgfqpoint{3.505455in}{4.224000in}}%
\pgfpathlineto{\pgfqpoint{5.760000in}{4.224000in}}%
\pgfusepath{stroke}%
\end{pgfscope}%
\begin{pgfscope}%
\definecolor{textcolor}{rgb}{0.000000,0.000000,0.000000}%
\pgfsetstrokecolor{textcolor}%
\pgfsetfillcolor{textcolor}%
\pgftext[x=4.632727in,y=4.307333in,,base]{\color{textcolor}\sffamily\fontsize{12.000000}{14.400000}\selectfont \(\displaystyle v_b(t)\)}%
\end{pgfscope}%
\end{pgfpicture}%
\makeatother%
\endgroup%

                    \end{center}
                    \caption{Grafici izračunatih funkcija $v_a(t)$ i $v_b(t)$ u prvom zadatku pod b.}
                    \label{MPL1b}
                \end{figure}
                \begin{figure}[H]
                    \centering
                    \subfloat[Kolo za simulaciju prvog zadatka pod a.]{\includegraphics[width=0.4\textwidth]{DZ1-1a.png}}
                    \hfill
                    \subfloat[Kolo za simulaciju prvog zadatka pod b.]{\includegraphics[width=0.4\textwidth]{DZ1-1b.png}}
                    \caption{Šeme za verifikaciju prvog zadatka.}
                    \label{PSpice1}
                \end{figure}
    
        \newpage
        \section{Zadatak 2}
            \subsection{Postavka}
                \begin{enumerate}[label=(\alph*)]
                    \item Izračunati frekvenciju oscilovanja kola i nacrtati vremenske dijagrame napona na izlazu idealnog operacionog pojačavača ($v_{IOP}$) i na izlazu komparatora K ($v_K$) u ustaljenom režimu, slika 2, u trajanju od 3 periode oscilovanja. Trajanje kvazistabilnog stanja monostabilnog multivibratora (MMV) iznosi $T_D = 10\mu s$, i za to vreme je izlaz MMV na visokom (5V) logičkom nivou. MMV se trigeruje silaznom ivicom signala sa izlaza komparatora. Prekidač je idealan. Kada je na izlazu MMV visok logički nivo prekidač je zatvoren, a kada je na izlazu MMV nizak logički nivo (0V) prekidač je otvoren. Poznato je i: $C = 10nF$, $V_R = 3V$, $I_R = 10mA$, $i_U = 2mA$.
                    \item Korišćenjem CMOS logičkih kola po izboru realizovati MMV iz tačke a) ako su na raspolaganju otpornici od $10k\Omega$ i kondenzatori po izboru.
                    \item Upotrebom PSpice simulacije verifikovati rezultate tačke a). Za integrator koristiti idealan operacioni pojačavač (OPAMP komponenta) sa napajanjem $\pm 12V$, dok za komparator treba koristiti idealan operacioni pojačavač sa napajanjem 0V i 5V. Za logička kola u MMV treba koristiti HC familiju, a kao prekidač koristiti naponski kontrolisani prekidač SBreak.
                \end{enumerate}

            \subsection{Rešenje}
                \subsubsection{Perioda kola}
                    Pretpostavićemo da je na početku kondenzator prazan, tako da kroz njega ne protiče struja, i da monostabilni multivibrator (MMV) izbacuje napon vrednosti logičke nule. To znači da je prekidač otvoren, i da je jedina stvar koja utiče na napon na kondezatoru $C$ struja $i_U$. Napon na kondenzatoru se menja linearno po formuli:
                    \begin{equation}
                        \label{Vc}
                        \frac{I_0}{C} t
                    \end{equation}
                    Dobijamo da napon $V_{IOP}$ uskoro dostiže vrednost 3V, i izlaz komparatora se menja sa logičke jedinice na logičku nulu, čime uzrokuje da MMV izbaci napon logičke jedinice u trajanju $T_D$. Kada je prekidač spušten, na napon na kondenzatoru $C$ utiče struja $I_R - i_U$, i to tokom trajanja celog kvazistabilnog stanja. Na osnovu jedinačine \eqref{Vc} dobijamo da napon $V_{IOP}$ opadne za vrednost:
                    \begin{equation}
                        \label{deltaIOP}
                        \Delta V_{IOP} = \frac{I_R - i_U}{C} T_D = \frac{10mA - 2mA}{10nF} 10\mu s = 8V
                    \end{equation}
                    Kada se završi kvazistabilno stanje MMV, prekidač se otvara i na napon na kondenzatoru ponovo utiče struja $i_U$. Na osnovu jednačina \eqref{Vc} i \eqref{deltaIOP} možemo izračunati vreme potrebno da napon na kondenzatoru ponovo dostigne 3V i time završi jednu oscilaciju:
                    $$\frac{i_U}{C} T_U = \Delta V_{IOP}$$
                    \begin{equation}
                        \label{deltaTIOP}
                        T_U = \frac{\Delta V_{IOP} C}{i_U} = \frac{8V \cdot 10nF}{2mA} = 40\mu s
                    \end{equation}
                    Iz jednačine \eqref{deltaTIOP} i vremena trajanja kvazistabilnog stanja $T_D$ dobijamo periodu ovog kola:
                    \begin{equation}
                        \label{T}
                        T = T_D + T_U = 50\mu s
                    \end{equation}
                    Na osnovu \eqref{T} dobijamo traženu frekvenciju:
                    $$f = \frac{1}{T} = 20kHz$$
                    Na osnovu gornje analize vidimo da se napon $V_{IOP}$ menja po formuli:
                    \begin{equation}
                        \label{VIOP}
                        V_{IOP} = \left\{\begin{array}{ll}
                            3V - \frac{I_R - i_U}{C} t & t \in (nT, nT + T_D] \\
                            -5V + \frac{i_U}{C} t & t \in (nT + T_D, (n+1) T]
                        \end{array}\right.
                        n \in \mathbb{N}
                    \end{equation}
                    Napon $V_K$ ima vrednost logičke nule za jako mali interval vremena kada $V_{IOP}$ dostigne 3V, pa vidimo da je za njega formula:
                    % $$V_K = 5V \sum_{k = -\infty}^{+\infty} (1 - \delta(t - nT - T_D))$$
                    % gde je $\delta$ Dirakov impuls.
                    \begin{equation}
                        \label{VK}
                        V_K = \left\{\begin{array}{ll}
                            0 & t = nT + T_D \\
                            5V & t \neq nT + T_D
                        \end{array}\right.
                        n \in \mathbb{N}
                    \end{equation}

                \subsubsection{Monostabilni multivibrator}
                    \begin{figure}[H]
                        \centering
                        \includegraphics[width=\textwidth,height=\textheight,keepaspectratio]{DZ1-2b.png}
                        \caption{Šema multivibratora iz zadatka.}
                        \label{PSpice2b}
                    \end{figure}
                    Šema monostabilnog multivibratora u ovom zadatku je slična šemi multivibratora u zadatku 20. sa drugih vežbi. Dok se na ovaj multivibrator dovodi napon logičke jedinice, na izlazu NI kola je logička nula, pa je na kondenzatoru napon logičke jedinice (na obe strane je logička nula). Kada napon na ulazu padne na logičku nulu, izlaz NI kola se promeni na logičku jedinicu, pa je potencijal na drugoj strani kondenzatora logička nula, i zbog toga struja protiče kroz generator napona u kondenzator. Napon na kondenzatoru se onda linearno povećava po jednačini iz \eqref{Vc}. Tako se potencijal povećava dok NI kolo ne detektuje napon logičke jedinice na ulazu, i za to vreme je na izlazu stanje logičke jedinice. To znači da za vreme $T_D$ napon mora da se poveća od logičke nule do napona praga, pa dobijamo sledeću jednačinu:
                    $$\frac{I_0}{C} T_D = V_{TL} - V(0)$$
                    $$\frac{\frac{5V}{10k\Omega}}{C} 10\mu s = 1V - 0V$$
                    $$C = 5nF$$
                \subsubsection{Simulacija}
                    \begin{figure}[H]
                        \centering
                        \includegraphics[width=\textwidth,height=\textheight,keepaspectratio]{DZ1-2c.png}
                        \caption{Šema simulacije zadatka.}
                        \label{PSpice2c}
                    \end{figure}
                    Zbog neidealnosti komponenti, u simulaciji je kondenzator promenjen na 6nF, jer je inače napon umesto sa 3V na -5V padao na oko -3.5V. Šema sa slike \ref{PSpice2c} daje rezultate sa slika \ref{PSpice2} i \ref{MPL2}, pa su ovime verifikovane izračunate vrednosti.
                    \begin{figure}[H]
                        \centering
                        \includegraphics[width=\textwidth,height=\textheight,keepaspectratio]{DZ1-2.pdf}
                        \caption{Rezultati simulacije drugog zadatka.}
                        \label{PSpice2}
                    \end{figure}
                    \begin{figure}[H]
                        \begin{center}
                            \input{DZ1-2.pgf}
                        \end{center}
                        \caption{Grafici izračunatih funkcija $V_{IOP}(t)$ i $V_K(t)$.}
                        \label{MPL2}
                    \end{figure}
    
\end{document}
